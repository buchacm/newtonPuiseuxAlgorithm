\documentclass[a4paper,draft]{amsart}
\usepackage[a4paper,margin=25mm]{geometry}
\usepackage[utf8]{inputenc}
\usepackage{mathtools}
\usepackage{amsfonts}
\usepackage{amsthm}
\usepackage{amsmath}
\usepackage{amssymb}
\usepackage{tikz}
\usepackage{hyperref}

\newtheorem{Problem}{Problem}
\newtheorem{Question}{Question}
\newtheorem{Remark}{Remark}
\newtheorem{Theorem}{Theorem}
\newtheorem{Proposition}{Proposition}
\newtheorem{Lemma}{Lemma}
\newtheorem{Conjecture}{Conjecture}
\newtheorem{Corollary}{Corollary}
\newtheorem{Algorithm}{Algorithm}

\def\clap#1{\hbox to0pt{\hss#1\hss}}
\def\eatspace#1{#1}
\def\step#1#2{\par\kern1pt\dimen44=#2em\advance\dimen44 1.67em\hangindent\dimen44\hangafter=1\noindent\rlap{\small#1}\kern\dimen44\relax\eatspace}
\let\set\mathbb
\def\<#1>{\langle#1\rangle}
\def\K{\set K}
\def\supp{\operatorname{supp}}

\theoremstyle{definition}
\newtheorem{Definition}{Definition}
\newtheorem{Example}{Example}
\newtheorem{Notation}{Notation}

\makeatletter
\def\testb#1{\testb@i#1,,\@nil}%
\def\testb@i#1,#2,#3\@nil{%
  \draw[->] (O) --++(#1);
  \ifx\relax#2\relax\else\testb@i#2,#3\@nil\fi}
\makeatother
\newcommand{\makediag}[1]{
    \coordinate (O) at (0,0); \coordinate (N) at (0,0.8);
    \coordinate (NE) at (0.8,0.8); \coordinate (E) at (0.8,0);
    \coordinate (SE) at (0.8,-0.8); \coordinate (S) at (0,-0.8);
    \coordinate (SW) at (-0.8,-0.8);\coordinate (W) at (-0.8,0);
    \coordinate (NW) at (-0.8,0.8); \coordinate (B1) at (1.2,1.2);
    \coordinate (B2) at (-1.2,-1.2);
    \testb{#1}
}
\newcommand{\diagr}[1]{
  \begin{tikzpicture}[baseline=-3pt,scale=0.3]\makediag{#1}\end{tikzpicture}
}

\DeclareMathOperator{\sep}{sep}
\DeclareMathOperator{\ind}{ind}
\DeclareMathOperator{\Gal}{Gal}
\DeclarePairedDelimiter\ceil{\lceil}{\rceil}
\DeclarePairedDelimiter\floor{\lfloor}{\rfloor}
\def\O{\mathrm{O}}
\def\sgn{\operatorname{sgn}}
\def\diag{\operatorname{diag}}
\def\clap#1{\hbox to0pt{\hss#1\hss}}
\let\ideal\unlhd
\def\bigtimes{\mathop{\hbox{\Large$\times$}}}

\title[Effective arithmetic for multivariate algebraic series]{Effective arithmetic for multivariate algebraic series}

%\date{}
%\author{Manfred Buchacher}

\begin{document}

\maketitle

\begin{center}
\begin{tabular}{@{}c@{}}
    Manfred Buchacher \\
    \normalsize manfredi.buchacher@gmail.com
  \end{tabular}%
  \end{center}

%==== begin general summary original story ==== 
%
%   startpunkt: funktionalgleichung; weitere gleichungen fuehren auf den ``orbit''
%
%   frage: lassen sich die gleichungen linear-kombinieren, so dass die sections wegfallen?
%   problem: wo leben diese gleichungen ueberhaupt? brauchen einen koerper, der sie alle
%   enthaelt, damit die linear-kombination sinnvoll ist. und algorithmisch muss es sein.
%
%   idee: konstruiere den splitting field der algfuns, die im orbit vorkommen und rechne dort.
%   das geht in einem ``formalen'' koerper der form K[y]/I ohne reihen. Damit laesst sich die
%   section-elimination-frage definitiv beantworten.
%
%   frage: wenn die sections eliminiert sind (nehmen wir an es ging), laesst sich der Koerper
%   K[y]/I in einen Koerper C((x)) so einbetten, dass die positive part extraction der
%   orbitsummengleichung die form F(x,y)=.. hat?
%
%   idee: berechne mit newton-puiseux fuer jede totalordnung alle reihenentwicklungen des
%   erzeugers von K[y]/I.
%
%   problem: es ist fuer jede solche einbettung zu bestimmen, ob ein term alg*F(alg,alg),
%   der in der orbitsumme vorkommt, einer Reihe entspricht, die keinen Term mit nur positiven
%   exponenten enthaelt.
%
%   idee: bestimme eine moeglichst gute abschaetzung der supporte der beteiligten algfuns
%   und wende dann Thm 17 an.
%
%   frage: wie bekommt man eine ``moeglichst gute'' abschaetzung?
%
%   antwort: das wissen wir nicht. wir haben eine vermutung: die bestmoegliche abschaetzung
%   ergibt sich aus der konvexen Huelle des Supports der Entwicklung der algfun in C((x)).
%
%   frage: wie rechnet man diese aus?
%
%   antwort: wende newton-puiseux auf das minpoly der algfun an, mit allen moeglichen ordnungen.
%   fuer jede ordnung liefert das mehrere loesungen, von denen hoechstens eine unsere algfun ist.
%   wir muessen entscheiden ob und wenn ja welche. 
%
%   frage: wie entscheidet man gleichheit von reihen, die evtl. a priori in verschiedenen
%   koerpern leben?
%
%   antwort: mit positive part extraction pruefen, ob die supports im gleichen kegel liegen
%   (details inspired by mexican paper), dann check nichtgleichheit aller conjugates bis
%   nur ein kandidat uebrig bleibt, wenn der im selben koerper lebt, muss er gleich sein.
%
%   conjecture: diesen aufwand muss man gar nicht betreiben. wir erwarten folgendes: gegeben
%   f, g mit initvals bzgl zweier ordnungen w1, w2, entwickle g weiter bis die ersten terme
%   von f bzgl w1 erreicht werden. ist dann f = init + Cone1 und g = extendedinit + Cone2,
%   so muss Cone2 subseteq Cone1 gelten, wobei Cone1 und Cone2 die Kegel von Macdonald sind.
%   (Tatsaechlich ist die Vermutung, dass die Kegel nach weit genuger abwicklung minimal sind.
%   das sollte obiges implizieren.)
%   
%==== end general summary original story ====
%
%==== begin new structure (attempt) ===
%
%0. Introduction
%-- motivating both parts of the paper. 
%
%1. Algebraische Funktionen in several variables (preliminaries): 
%-- wo leben ihre series expansions (in welchen koerper/ringen?)
%-- wie bestimmt man ihre anfangsterme (gegeben eine termordnung)
%-- mexican papers reviews.
%-- kauers-french paper ueber residue extraction with creative telescoping fuer positive part extraction. 
%
%2. Konstruktive Arithmetik
%-- wollen: gleichheitstest fuer algebraische funktionen, die durch minpoly, initvals und ggf verschiedene(!) termordnungen spezifiziert sind. (main result)
% o wenn f und g gleich sind, muss ihr support in supp(f) cap supp(g) liegen.
% o bestimme f0 := f eingeschraenkt auf supp(g) (generalized positive part) [oder umgekehrt] und betrachte f0 - f
% o das ist zumindest D-finite.
% o gap theorem like result gives number of initial values to be checked to confirm (or disprove) that f0-f = 0.
%-- koennen dann: auch + und * effektiv durchfuehren (ausschlussprinzip [very clever: nachdem man mit =-test die vielen verschiedenen reihenloesungen des minpolys von f+g zusammengefasst hat, funktioniert das ausschlussprinzip erst, weil erst dann gibt es nur noch eine einzige reihe, die gleich f+g ist. vorher gibt es sie mehrmals] )
%-- auch useful for: gute abschaetzung des supports einer solchen reihe (standard algorithmus gibt i.a. einen zu konservativen kegel).
% o bestimmen aller beschraenkten faces der konvexen huelle geht immer (verwendet =-test)
% o wenn der output von macdonald garantiert ultimately minimal ist, dann koennen wir sogar die konvexe huelle bestimmen (incl infinite faces)
% o abgrenzung zum ``minimal cone'' paper der mexikaner
%-- conjecture
%-- unterscheide zw alg closure of C(x,y) and a finite extension of C(x,y); es geht nicht nur um interpretationen einzelner funktionen als reihen, sondern vielmehr um eine einbettung von K[theta]/I in ein C((x)) (mit homo und so). // just bla-bla to create context.
%
%3. Application: the orbit sum method
%-- kombinatorisches problem; funktionalgleichung; weitere gleichungen fuehren auf den ``orbit''
%-- frage: lassen sich die gleichungen linear-kombinieren, so dass die sections wegfallen?
%   problem: wo leben diese gleichungen ueberhaupt? brauchen einen koerper, der sie alle
%   enthaelt, damit die linear-kombination sinnvoll ist. und algorithmisch muss es sein.
%-- idee: konstruiere den splitting field der algfuns, die im orbit vorkommen und rechne dort.
%   das geht in einem ``formalen'' koerper der form K[y]/I ohne reihen. Damit laesst sich die
%   section-elimination-frage definitiv beantworten.
%-- frage: wenn die sections eliminiert sind (nehmen wir an es ging), laesst sich der Koerper
%   K[y]/I in einen Koerper C((x)) so einbetten, dass die positive part extraction der
%   orbitsummengleichung die form F(x,y)=.. hat?
%-- idee: berechne mit macdonald fuer jede totalordnung alle reihenentwicklungen des
%   erzeugers von K[y]/I, und verwende die theorie von oben, um gute abschaetzungen fuer die
%   Traeger der algfuns zu bestimmen, dann pruefe mit Thm. 17, ob ausgeschlossen werden kann,
%   dass alg*F(alg,alg) einen Beitrag zum ersten orthanten liefert. falls ja, pech (keine Entscheidung!)
%   falls nein, happy (success!)
%
%==== end new structure (attempt) ===
%
%\section{Introduction}
%
%% series expansions of rational functions in one variable
%To every rational function in one variable~$x$, we can associate at most two series objects in~$x$:
%a series in ascending powers and a series in descending powers.
%For example, the rational function $\frac1{1-x}$ can be expanded as $1+x+x^2+\cdots$ or
%as $-x^{-1}-x^{-2}-x^{-3}+\cdots$ (note that $\frac1{1-x}=-\frac{1/x}{1-1/x}$).
%The two expansions coincide if and only if the rational function is actually a polynomial.
%
%% series expansions of rational functions in several variables
%The case of several variables is more diverse.
%A rational function in two variables $x,y$ can in general be written in many different ways as a
%series in~$x,y$.
%The exponent vectors $(i,j)$ of all the terms $cx^iy^j$ with $c\neq0$ appearing in such a series
%belong to a certain cone rooted at one of the terms of the series.
%For example, the rational function $\frac{x+2y}{1+x+y}$ admits the following four series expansions:
%\begin{center}
%\begin{tikzpicture}[scale=.2]
%  \fill[gray] (0,0)--(-6,6)--(-6,0)--cycle;
%  \draw[->](-6.5,0)--(6.5,0);
%  \draw[->](0,-6.5)--(0,6.5);
%  \foreach \x/\y in {-6/0, -6/1, -6/2, -6/3, -6/4, -6/5, -5/0, -5/1, -5/2, -5/3, -5/4, -4/0, -4/1, -4/2, -4/3, -3/0, -3/1, -3/2, -2/0, -2/1, -1/0, 0/0}
%  \fill (\x,\y) circle(5pt);
%  \draw (0,-7) node[below] {\parbox{2.8cm}{\centering\scriptsize $f_1=
%      1-x^{-1}+x^{-2}+x^{-1}y-x^{-3}+x^{-4}-x^{-3}y-x^{-2}y^2-x^{-5}+2x^{-4}y+x^{-3}y^2+x^{-6}-3x^{-5}y+x^{-3}y^3%-x^{-7}+4x^{-6}y
%      +\cdots$}};
%\end{tikzpicture}\hfil
%\begin{tikzpicture}[scale=.2]
%  \fill[gray] (-5,6)--(1,0)--(6,0)--(6,6)--cycle; 
%  \draw[->](-6.5,0)--(6.5,0);
%  \draw[->](0,-6.5)--(0,6.5);
%  \foreach \x/\y in { 0/1, 0/2, 0/3, 0/4, 0/5, 0/6, 1/0, 1/1, 1/2, 1/3, 1/4, 1/5, 1/6, 2/0, 2/1, 2/2, 2/3, 2/4, 2/5, 2/6, 3/0, 3/1, 3/2, 3/3, 3/4, 3/5, 3/6, 4/0, 4/1, 4/2, 4/3, 4/4, 4/5, 4/6, 5/0, 5/1, 5/2, 5/3, 5/4, 5/5, 5/6, 6/0, 6/1, 6/2, 6/3, 6/4, 6/5, 6/6 }
%  \fill (\x,\y) circle(5pt);
%  \draw (0,-7) node[below] {\parbox{2.8cm}{\centering\scriptsize $f_2=
%      2y+x-2y^2-3xy+2y^3-x^2+5xy^2-2y^4+4x^2y-7xy^3+2y^5+x^3-9x^2y^2+9xy^4-2y^6-5x^3y
%      +\cdots$}};
%\end{tikzpicture}\hfil
%\begin{tikzpicture}[scale=.2]
%  \fill[gray] (0,6)--(0,1)--(6,-5)--(6,6)--cycle; 
%  \draw[->](-6.5,0)--(6.5,0);
%  \draw[->](0,-6.5)--(0,6.5);
%  \foreach \x/\y in { 0/1, 0/2, 0/3, 0/4, 0/5, 0/6, 1/0, 1/1, 1/2, 1/3, 1/4, 1/5, 1/6, 2/0, 2/1, 2/2, 2/3, 2/4, 2/5, 2/6, 3/0, 3/1, 3/2, 3/3, 3/4, 3/5, 3/6, 4/0, 4/1, 4/2, 4/3, 4/4, 4/5, 4/6, 5/0, 5/1, 5/2, 5/3, 5/4, 5/5, 5/6, 6/0, 6/1, 6/2, 6/3, 6/4, 6/5, 6/6 }
%  \fill (\x,\y) circle(5pt);
%  \draw (0,-7) node[below] {\parbox{2.8cm}{\centering\scriptsize $f_3=
%      x-x^2+2y+x^3-3xy-x^4+4x^2y-2y^2+x^5-5x^3y+5xy^2-x^6+6x^4y-9x^2y^2+2y^3+x^7-7x^5y
%      +\cdots$}};
%\end{tikzpicture}\hfil
%\begin{tikzpicture}[scale=.2]
%  \fill[gray] (0,0)--(6,-6)--(0,-6)--cycle;
%  \draw[->](-6.5,0)--(6.5,0);
%  \draw[->](0,-6.5)--(0,6.5);
%  \foreach \x/\y in { 0/-6, 0/-5, 0/-4, 0/-3, 0/-2, 0/-1, 0/0, 1/-6, 1/-5, 1/-4, 1/-3, 1/-2, 2/-6, 2/-5, 2/-4, 2/-3, 3/-6, 3/-5, 3/-4, 4/-6, 4/-5, 5/-6  }
%  \fill (\x,\y) circle(5pt);
%  \draw (0,-7) node[below] {\parbox{2.8cm}{\centering\scriptsize $f_4=
%      2-2y^{-1}+2y^{-2}-xy^{-1}-2y^{-3}+3xy^{-2}+2y^{-4}-5xy^{-3}+x^2y^{-2}-2y^{-5}+7xy^{-4}%-4x^2y^{-3}+2y^{-6}-9xy^{-5}+9x^2y^{-4}
%      +\cdots$}};
%\end{tikzpicture}
%\end{center}
%Observe that $f_2$ and $f_3$ are in fact the same series, just written down differently, while
%$f_1$ and $f_4$ are really different. The series $f_2$ and $f_3$ are equal because there is a
%ring of series which contains both of them, and a rational function cannot have more than one
%expansion in the same ring of series.
%
%% series expansions of algebraic functions in several variables
%Algebraic functions may have more than one expansion in a fixed ring. For example, the polynomial
%$Y^2-xY-1$ has the two roots $-1+\tfrac12x-\tfrac18x^2+\cdots$ and $1+\tfrac12x+\tfrac18x^2+\cdots$
%in the formal power series ring $\set Q[[x]]$.
%Algebraic functions in several variables can again be developed into series with supports in
%cones. Like for rational functions, there are in general many different directions, and unlike
%for rational function, we may now have more than one expansion for each direction. 
%For algebraic functions, it is therefore not so easy to recognize whether some of their
%expansions are in fact identical.
%
%% contribution: decision procedure for equality
%In this paper, we propose an algorithm for deciding this question.
%More precisely, our algorithm takes as input two multivariate algebraic series specified in
%terms of a minimal polynomial, a line-free cone containing the support of the series, and
%sufficiently many initial terms to distinguish the series from all other series solutions
%of the minimal polynomial whose support belongs to the specified cone.
%It returns true or false depending on whether or not the two input series specified
%in this way are equal or not.
%
%% small applications
%With the help of an algorithm for deciding equality, we are able to constructively perform
%arithmetic in an algebraic closure of a multivariate rational function field. This means
%that we obtain algorithms which from any two algebraic series as specified above can
%construct such a representation for their sum and their product.
%As another application, our equality test is useful for analyzing the support of an
%algebraic series. The data structure carries along a cone to which the support belongs,
%but this cone may be larger than necessary. The idea is that if we recognize that
%two series whose specifications contain different cones are actually equal, then their
%supports must actually belong to the intersection of these two cones. If we are lucky,
%we can find a smaller cone in this way.
%
%% main application
%Our original motivation arose in the context of counting restricted lattice walks.
%The combinatorial definition of these objects translates into a certain functional
%equation for the generating function, and there is a method known as the orbit sum
%method for solving such equations. In its original version, the method requires
%interpreting multivariate rational functions as series. More recent versions of the
%orbit sum method developed for counting more complicated types of restricted lattice
%walks require the interpretation of multivariate algebraic functions as series, and
%applying the orbit sum method in this setting requires testing such series for being
%equal and understanding their supports. 
%
%% roadmap
%A more detailed account on the combinatorial problem and its solution via the orbit
%sum method is given in Sect.~\ref{sec:orbit}. In Sect.~\ref{sec:alg}, we present our decision
%procedure. This part is independent of our combinatorial application, but it depends
%on some known results about multivariate series that we summarize in Sect.~\ref{sec:prelim}.
%
%\section{Preliminaries}\label{sec:prelim}
%
%Series expansions of multivariate algebraic functions were studied by MacDonald~\cite{..},
%who generalized the classical Newton-Puiseux algorithm~\cite{..,..,..} from the univariate
%case to the case of several variables. In this setting, we are interested in infinite series
%\[
%  a = \sum_{i_1,\dots,i_n\in\set Q} a_{i_1,\dots,i_n} x_1^{i_1}\cdots x_n^{i_n}
%\]
%with coefficients $a_{i_1,\dots,i_n}$ in some field~$K$.
%The \emph{support} of such a series is defined as the set of exponent vectors of all the
%terms which have a nonzero coefficient:
%\[
%  \supp(a) := \{ (i_1,\dots,i_n)\in\set Q : a_{i_1,\dots,i_n}\neq0 \}.
%\]
%Like in the univariate case, the exponents are allowed to be rational, but they should
%have a finite common denominator, so we restrict the attention to series $a$ for which
%there exists a $k\in\set Z$ with $k\supp(a)\in\set Z^n$.
%
%In order to be able to multiply series, we need to restrict their supports further. 
%A cone is a subset of $\set R^n$ which closed under multiplication by nonnegative real numbers.
%A cone is convex if it is also closed under addition.
%It is called strongly convex or line-free or pointed if it does not contain a line.
%It is called polyhedral if there are $v_1,\dots,v_n\in\set R^n$ such the elements of the
%cone are precisely the linear combinations of $v_1,\dots,v_n$ with nonnegative real coefficients.
%It is called rational polyhedral if the $v_1,\dots,v_n$ can be chosen from~$\set Q^n$.
%In this paper, we will only consider strongly convex rational polyhedral cones, and we will
%simply call them cones from now on.
%
%For a cone $C\subseteq\set R^n$ and a field~$K$, MacDonald defines $K_C[[x_1,\dots,x_n]]$ as the
%ring of all series $a$ such that $\supp(a)\subseteq C\cap\frac1k\set Z^n$ for some $k\in\set Z$.
%It turns out that $K_C[[x_1,\dots,x_n]]$ is indeed a ring (see also \cite{..}). It is however
%not a field. In order to get a field, we impose an order on the set of all terms $x_1^{e_1}\cdots x_n^{e_n}$.
%The order must be compatible with multiplication in the sense that we want
%$\sigma\leq\tau\Rightarrow\rho\sigma\leq\rho\tau$ to be true for all terms $\sigma,\rho,\tau$.
%A quick way to obtain such an order is to pick a vector $w=(w_1,\dots,w_n)\in\set R^n$
%whose coordinates are linearly independent over $\set Q$ and then to define
%\[
%x_1^{u_1}\cdots x_n^{u_n}
%\leq
%x_1^{v_1}\cdots x_n^{v_n}
%\iff
%\sum_{i=1}^n w_iu_i \leq \sum_{i=1}^n w_iv_i.
%\]
%Geometrically, the vector $w$ defines a halfspace $H=\{ z\in\set R^n : z\cdot w\geq0 \}\subseteq\set R^n$,
%and $\sigma\leq\tau$ means that there is a translated
%copy of $H$ which contains the exponent vector of $\tau$ but not the exponent vector of~$\sigma$.
%\begin{center}
%  \begin{tikzpicture}[scale=.2]
%    \fill[lightgray] (-4,6)--(6,6)--(6,-6)--(4,-6)--cycle;
%    \draw[->](-6.5,0)--(6.5,0);
%    \draw[->](0,-6.5)--(0,6.5);
%    \clip (-6,-6) rectangle (6,6);
%    \draw[thick] (-4,6)--(4,-6) (3,4.5) node {$H$};
%    \draw[thick,->] (0,0)--(1.8,1.2) node[right] {$w$};
%    \draw[xshift=-2.5cm] (-4,6)--(4,-6)
%       (-3,3) node {$\bullet$} node[below] {$\sigma$}
%       (0,1) node {$\bullet$} node[above] {$\tau$} ;
%  \end{tikzpicture}
%\end{center}
%We say that a cone $C$ is compatible with an order $\leq$ if $C$ has
%a minimal element w.r.t.~$\leq$. Geometrically, this means that $C$ is contained in the
%halfspace~$H$. If $\mathcal{C}$ denotes the set of all cones that are compatible with
%a fixed term order~$\leq$, then we set
%\[
%  K_\leq((x_1,\dots,x_n)) := \bigcup_{e\in\set Q^n}\bigcup_{C\in\mathcal{C}} x^e K_C[[x_1,\dots,x_n]].
%\]
%The elements of $K_\leq((x_1,\dots,x_n))$ are thus all the series whose support is contained
%in a shifted copy of a cone that is compatible with~$\leq$.
%It turns out that $K_\leq((x_1,\dots,x_n))$ is a field (cf.~\cite[Thm. ??]{..}).
%
%The main result of MacDonald (his Thm.~3.6, slightly rephrased) asserts the following:
%if $K$ is an algebraically closed field, $\leq$ is a term order, and $p\in
%K[x_1,\dots,x_n][y]$ is a squarefree polynomial of degree~$d>0$ in~$y$, then
%$K_\leq((x_1,\dots,x_n))$ contains $d$ distinct roots of~$p$. Moreover, there is
%an algorithm which for any given $N\in\set N$ can compute the $N$ smallest terms
%(w.r.t. $\leq$) of each of these roots. The algorithm also returns a cone which
%is compatible with $\leq$ and has the property that the exponent vectors of all
%the remaining terms belong to the shifted copy of this cone rooted at the exponent
%vector of the last computed term.
%\begin{center}
%  \begin{tikzpicture}[scale=.2]
%    \fill[lightgray] (-4,6)--(6,6)--(6,-6)--(4,-6)--cycle;
%    \fill[gray] (3,2)--(6,-1)--(6,6)--(3,6)--cycle;
%    \draw[->](-6.5,0)--(6.5,0);
%    \draw[->](0,-6.5)--(0,6.5);
%    \clip (-6,-6) rectangle (6,6);
%    \foreach \x/\y in {0/1, 0/2, 0/-1, 2/3, -1/2, -1/1, 1/1, 1/-2, 1/-3, 2/0, 3/0, 3/2} \draw (\x,\y) node {$\bullet$};
%  \end{tikzpicture}
%\end{center}
%
%// connection to recent mexican papers
%
%An element of $K_\leq((x_1,\dots,x_n))$ is called \emph{algebraic} if it is a root of
%a polynomial $p\in K[x_1,\dots,x_n][y]\setminus\{0\}$, otherwise transcendental. If
%$a\in K_\leq((x_1,\dots,x_n))$ is algebraic, then for every $i\in\{1,\dots,n\}$ it
%also satisfies a linear differential equation of the form 
%\[
%p_0 a + p_1 \frac{\partial}{\partial x_i} a + \cdots + p_r \frac{\partial^r}{\partial x_i^r} a = 0
%\]
%with $p_0,\dots,p_r\in K[x_1,\dots,x_n]$ not all zero (Abel's theorem~\cite{..}).
%Conversely, not every series $a\in K_\leq((x_1,\dots,x_n))$ which satisfies such
%a system of $n$ linear differential equations is algebraic. A series satisfying
%such a system is called \emph{D-finite}~\cite{..,..,..}.
%
%D-finiteness is a robust concept in the sense that there are many operations which
%can be applied to a D-finite series without destroying its D-finiteness. For example,
%if $a\in K_\leq((x_1,\dots,x_n))$ is D-finite, then so is its \emph{positive part,}
%defined by
%\[
%  [x_1^>\cdots x_n^>]
%  \sum_{i_1,\dots,i_n\in\set Q} a_{i_1,\dots,i_n} x_1^{i_1}\cdots x_n^{i_n}
%  := \sum_{i_1,\dots,i_n>0} a_{i_1,\dots,i_n} x_1^{i_1}\cdots x_n^{i_n}.
%\]
%\begin{center}
%  \begin{tikzpicture}[scale=.2]
%    \fill[gray] (-2,-3)--(-5,6)--(4,6)--(-2,-3)--cycle;
%    \draw[->](-6.5,0)--(6.5,0);
%    \draw[->](0,-6.5)--(0,6.5);
%    \draw (10,0) node {$\to$};
%    \begin{scope}[xshift=20cm]
%      \begin{scope}
%        \clip (-2,-3)--(-5,6)--(4,6)--(-2,-3)--cycle;
%        \foreach \x in {-7,-6.6,...,2} \draw (\x,-3)--(\x+1,6);
%      \end{scope}
%      \begin{scope}
%        \clip (.2,0) rectangle (6,6);
%        \fill[gray] (-2,-3)--(-5,6)--(4,6)--(-2,-3)--cycle;
%      \end{scope}
%      \draw[->](-6.5,0)--(6.5,0);
%      \draw[->](0,-6.5)--(0,6.5);
%    \end{scope}
%  \end{tikzpicture}
%\end{center}
%This was first shown by Lipshitz~\cite{..} and later~\cite{..} updated to a more
%modern algorithm for computing a system of differential equations for
%$[x_1^>\cdots x_n^>]a$ from a given system of differential equations for~$a$.
%These papers do not discuss fractional exponents, but since our rational exponents
%are required to have a finite common denominator, this does not make any difference.
%Moreover, the constructions are not limited to positive part extraction but also
%allow us to compute for any given D-finite series $a\in K_\leq((x_1,\dots,x_n))$
%(specified by a system of differential equations) and any given cone
%$C\subseteq\set R^n$ (specified by a set of rational generators) a system of
%differential equations for the series
%\[
%  [C] \sum_{i_1,\dots,i_n\in\set Q} a_{i_1,\dots,i_n} x_1^{i_1}\cdots x_n^{i_n}
%  := \sum_{(i_1,\dots,i_n)\in C\cap\set Q^n} a_{i_1,\dots,i_n} x_1^{i_1}\cdots x_n^{i_n}.
%\]
%
%Another operation which preserves D-finiteness is the composition with
%algebraic functions. We need only the following special case of this:
%if $a=a(x_1,\dots,x_n)\in K_\leq((x_1,\dots,x_n))$ is D-finite and
%$\tau_1,\dots,\tau_n$ are terms whose exponent vectors are linearly
%independent, then $a(\tau_1,\dots,\tau_n)$ is a D-finite series
%(possibly no longer belonging to $K_\leq((x_1,\dots,x_n))$ but to
%a field with some other term order). This
%feature allows us to turn arbitrary cones into more comfortable positions.
%\begin{center}
%  \begin{tikzpicture}[scale=.2]
%    \fill[gray] (2,3.5)--(-6,5.5)--(-6,-6)--(6,-6)--(6,-4)--cycle;
%    \draw[->](-6.5,0)--(6.5,0);
%    \draw[->](0,-6.5)--(0,6.5);
%    \draw (10,0) node {$\to$};
%    \begin{scope}[xshift=20cm]
%      \fill[gray] (-1,-2)--(-3,6)--(4,6)--cycle;
%      \draw[->](-6.5,0)--(6.5,0);
%      \draw[->](0,-6.5)--(0,6.5);
%    \end{scope}
%  \end{tikzpicture}
%\end{center}
%Note in particular that the support of an element $a\in K_\leq((x_1,\dots,x_n))$ can in general
%involve for each variable $x_i$ arbitrarily high as well as arbitrarily small exponents.
%With the above substitution, we can in particular always turn an obtuse cone into an acute cone.
%This is useful because such series can be viewed as univariate
%Puiseux series in one variable, say~$x_n$, with coefficients that are Puiseux polynomials
%in the remaining variables $x_1,\dots,x_{n-1}$. This in turn is useful because univariate
%D-finite series are somewhat easier to handle than multivariate ones. In particular, writing
%$a=\sum_{m\in\set Z} a_m(x_1,\dots,x_{n-1})x_n^{m/k}$, the differential equation for $a$
%w.r.t.\ $x_n$ can be translated into a linear recurrence equation
%\[
%  p_0 a_m + \cdots + p_s a_{m+s} = 0
%\]
%with coefficients $p_0,\dots,p_s\in K[x_1,\dots,x_{n-1},m]$, $p_s\neq0$, for the
%coefficients of~$a$ (cf. \cite{..}). Viewed as elements of $K(x_1,\dots,x_{n-1})[m]$, the polynomial
%$p_s$ can only have finitely many roots. This has important consequences.
%For example, it is impossible for the coefficient sequence $(a_m)_{m\in\set Z}$ to have
%arbitrarily large gaps.
%\begin{center}
%  \begin{tikzpicture}[scale=.2]
%    \fill[gray] (-4,-4.5)--(-3,6)--(5,6)--cycle;
%    \draw[->](-6.5,0)--(6.5,0);
%    \draw[->](0,-6.5)--(0,6.5);
%    \draw (-6,1)--(6,1) (-6,3)--(6,3);
%    \draw (6,2) node[right] {\rlap{$\leftarrow$ maximum gap size}};
%    \clip (-6,1) rectangle (6,3);
%    \fill[lightgray] (-4,-4.5)--(-3,6)--(5,6)--cycle;
%    \foreach\x in {-6.8,-6.5,...,6} \draw (\x,1)--(\x+.2,3);
%  \end{tikzpicture}
%\end{center}
%This observation was recently used in~\cite{..} in order to
%prove that certain elements of $K_\leq((x_1,\dots,x_n))$ cannot be D-finite (and
%hence in particular not be algebraic).
%Another consequence is that there is some finite and computable $s_{\max}\in\set Z$
%such that $a_m=0$ for all $m\leq s_{\max}$ implies $a_m=0$ for all $m\in\set Z$.
%We will exploit this feature in our equality test. 
%
%\section{Identity Testing}\label{sec:alg}
%
%\section{The Orbit Sum Method}\label{sec:orbit}
%
%
%==== old stuff ahead ===
%
%\newpage


\begin{abstract}
We explain how to encode an algebraic series by a finite amount of data, namely its minimal polynomial, a total order and its first terms with respect to this order, and how to do effective arithmetic on the level of these encodings. The reasoning is based on the Newton-Puiseux algorithm and an effective equality test for algebraic series. We also explain how the latter allows to derive information about the support of an algebraic series, e.g. how to compute the (finitely many) vertices and bounded faces of the convex hull of its support. 
\end{abstract}

\section{Overview}

Given a polynomial $p(x,y)$ in two variables $x$ and $y$ over an algebraically closed field~$\mathbb{K}$ of characteristic zero, the classical Newton-Puiseux algorithm allows to determine the first terms of a series $\phi$ over $\mathbb{K}$ for which~$p(x,\phi) = 0$. Finding a series solution of a polynomial equation is one and the most apparent aspect of the algorithm. However, it also allows to encode a series by a finite amount of data and to effectively perform operations such as plus and times on the level of these encodings. The Newton-Puiseux algorithm was introduced by Newton and analyzed by Puiseux in~\cite{puiseux1850recherches}, see also~\cite{brieskorn2012plane} for a modern presentation of it. It was generalized to multivariate, not necessarily bivariate polynomials over a field of characteristic zero in~\cite{MacDonald}, and studied in~\cite{saavedra2017mcdonald} for polynomials over a field of positive characteristic. While it is well-known how to compute effectively with univariate series that are algebraic, this is not the case for algebraic multivariate series. We explain the latter here and complement the discussion of the Newton-Puiseux algorithm presented in~\cite{MacDonald}. 
%We also explain how the effective equality test the effective arithmetic is based on allows 
%The effective computation with algebraic series is based on an effective equality test. 
%Proving that two series are not equal can in general be done by computing and comparing their initial terms. In the univariate setting this can also be used to show that two series are equal. However, proving that two multivariate series are the same requires not only the comparison of initial terms but also an estimate of their supports. 
We also show that the convex hull of the support of an algebraic series is a polyhedral set and explain how the equality test allows to compute its vertices and bounded faces. Supports of series were also studied in~\cite{aroca2019support,aroca2022minimal}, though in the more general context of series algebraic over a certain ring of series. The article comes with a Mathematica implementation of the Newton-Puiseux algorithm and a Mathematica notebook, which can be found on~https://github.com/buchacm/newtonPuiseuxAlgorithm.


%has only finitely many vertices, explain how to determine these vertices and how to compute for each vertex $v$ a line-free cone $C$ for which $\mathrm{supp}(\phi)\subseteq v + C$. We also discuss how the minimality of $C$ can be checked. 

%\paragraph{\textbf{The Orbit-Sum Method.}} We illustrate this aspect of the Newton-Puiseux algorithm in the context of enumerative combinatorics. We explain how to determine the generating function of lattice walks restricted to convex cones by solving partial discrete differential equations via a generalization and algorithmization of the orbit-sum method. The orbit-sum method is an algebraic version of the reflection-principle~\cite{?} that was introduced in~\cite{?} to solve linear partial discrete differential equations of order~$1$. 
%Its extension to equations of higher order was approached in~\cite{?}. We continue its generalization making use of the primitive element theorem, Gr\"{o}bner bases and the shape lemma, and the Newton-Puiseux algorithm. 

%\part{The Newton-Puiseux Algorithm}

\section{Preliminaries}\label{sec:prelim}

We begin with introducing the objects this article is about: multivariate algebraic series, and the Newton-Puiseux algorithm to constructively work with them.\\

Let $\mathbb{K}$ be an algebraically closed field of characteristic zero, denote by $\bold{x} = (x_1,\dots,x_n)$ a vector of variables, and write~$\bold{x}^I:=x_1^{i_1}\cdot \dots \cdot x_n^{i^n}$ for $I = (i_1,\dots,i_n)\in\mathbb{Q}^n$. A series $\phi$ in $\bold{x}$ over $\mathbb{K}$ is a formal sum 
\begin{equation*}
\phi = \sum_{I\in\mathbb{Q}^n} a_I \bold{x}^I
\end{equation*}
of terms in $\bold{x}$ whose coefficients $a_I$ are elements of $\mathbb{K}$.
Its support is defined by 
\begin{equation*}
\mathrm{supp}(\phi) = \{I\in\mathbb{Q}^n: a_I \neq 0 \},
\end{equation*}
and we will assume throughout that there is an integer $k\in\mathbb{Z}$, a line-free cone $C\subseteq\mathbb{R}^n$ and a vector~$v\in\mathbb{R}^n$ such that 
\begin{equation*}
\mathrm{supp}(\phi) \subseteq \left( v + C \right) \cap \frac{1}{k}\mathbb{Z}^n.
\end{equation*}
Without any restriction on their supports, the sum and product of two series is not well-defined. However, for any line-free convex cone $C\subseteq\mathbb{R}^n$ the set $\mathbb{K}_C[[x]]$ of series whose support is contained in $C$ is a ring with respect to addition and multiplication~\cite[Theorem 10]{monforte2013formal}. Yet, it is not a field~\cite[Theorem 12]{monforte2013formal}.

Any~$w\in\mathbb{R}^n$ whose components are linearly independent over $\mathbb{Q}$ defines a total order $\preceq$ on $\mathbb{Q}^n$ by
\begin{equation*}
\alpha \preceq \beta \quad :\Longleftrightarrow \quad \langle \alpha, w\rangle \leq \langle \beta, w\rangle,
\end{equation*}
where $\langle \alpha,w \rangle := \sum_{i=1}^n \alpha_i w_i$ for $\alpha,w\in\mathbb{R}^n$. As usual, we write $\alpha \prec \beta$ when $\alpha\preceq \beta$ and $\alpha \neq \beta$. It naturally extends to the set of terms in $\bold{x}$ by saying that $a \bold{x}^\alpha \preceq b \bold{x}^\beta$ when $\alpha \preceq \beta$. We define $\mathrm{lexp}_\preceq(\phi) := \max_\preceq \mathrm{supp}(\phi)$, and we write $\mathrm{lexp}_w(\phi)$ for it when $\preceq$ is induced by $w\in\mathbb{R}^n$.
 A cone $C\subseteq \mathbb{R}^n$ is compatible with a total order~$\preceq$ on $\mathbb{Q}^n$ when $C\cap \mathbb{Q}^n$ has a maximal element with respect to it. Given a total order~$\preceq$, let $\mathcal{C}$ be the set of cones that are compatible with it. We write 
\begin{equation*}
\mathbb{K}_{\preceq}((\bold{x})):= \bigcup_{e\in\set Q^n}\bigcup_{C\in\mathcal{C}} x^e \mathbb{K}_C[[\bold{x}]]
\end{equation*} 
for the set of series whose support is contained in a shift of a cone compatible with $\preceq$. It is not only a ring but even a field~\cite[Theorem 15]{monforte2013formal}.\\ 

A series $\phi$ is said to be algebraic if there is a non-zero polynomial~$p\in\mathbb{K}[\bold{x},y]$ such that 
\begin{equation*}
p(\bold{x},\phi) = 0.
\end{equation*}
It is said to be D-finite if for every $i\in\{1,\dots,n\}$ there are $q_0,\dots,q_r\in\mathbb{K}[\bold{x}]$ such that 
\begin{equation*}
q_0 \phi + q_1 \frac{\partial}{\partial x_i}\phi + \dots + q_r \frac{\partial^r}{\partial x_i^r}\phi = 0.
\end{equation*}
Every algebraic series is D-finite~\cite[Theorem 6.1]{kauers2011concrete}, and as do algebraic series, D-finite series satisfy many closure properties. For instance, the sum $\phi_1 + \phi_2$ of two D-finite series $\phi_1$ and $\phi_2$ is D-finite~\cite[Theorem 7.2]{kauers2011concrete}, and so is the restriction 
\begin{equation*}
[\phi]_C(\bold{x}) := \sum_{I\in C\cap \mathbb{Z}^n}\left( [\bold{x}^I]\phi\right) \bold{x}^I.
\end{equation*}
of a D-finite series $\phi$ to a finitely generated rational convex cone $C$~\cite{bostan2017hypergeometric}.
These closure properties are effective in the sense that systems of differential equations for $\phi_1 + \phi_2$ and $[\phi]_C$ can be computed from the differential equations satisfied by $\phi_1$, $\phi_2$ and $\phi$. 

Having a univariate D-finite series $\phi(t) := \sum_{k\geq k_0} \phi_k t^k \in\mathbb{K}((t))$ and a differential equation satisfied by it, it is easy to check whether $\phi$ is identically zero. The differential equation for $\phi$ translates into a recurrence relation for its coefficients,
\begin{equation*}
p_0(k) \phi_k + p_1(k)\phi_{k+1} + \dots + p_r(k)\phi_{k+r} = 0, \quad p_0,\dots,p_r \in\mathbb{K}[k],
\end{equation*}
so that $\phi = 0$ if and only if $\phi_k = 0$ for finitely many $k$, the actual number depending on the largest integer root of $p_r$.\\
%$k=k_0,\dots,k_{\mathrm{max}}+r$, where $k_{\mathrm{max}}$ is the maximum of $0$ and the largest integer root of $p_r$.\\


Given $p\in\mathbb{K}[\bold{x},y]$ and a total order $\preceq$ on $\mathbb{Q}^n$ the Newton-Puiseux algorithm allows to determine the series solutions of $p(\bold{x},y) = 0$ in~$\mathbb{K}_\preceq((\bold{x}))$.
%The idea is to choose the first term $\phi_1$ of $\phi= \sum_{i=1}^\infty \phi_i$ such that the highest order terms in $p_1(x,\phi_1) := p(x,\phi_1)$ cancel, and to determine the next term $\phi_{k+1}$ from the terms $\phi_1,\dots,\phi_k$ already computed such that $\phi_{k+1} \prec \phi_k$ and the highest order terms in $p_{k+1}(x,\phi_{k+1}) := p(x,\sum_{i=1}^k \phi_i + \phi_{k+1})$ cancel. 
We collect a few definitions before we present it. The Newton polytope of~$p$ is the convex hull of the support of $p$,
 \begin{equation*}
\mathrm{NP}(p) := \mathrm{conv}(\mathrm{supp}(p)).
\end{equation*}
If $e$ is an edge of $\mathrm{NP}(p)$ that connects two vertices $v_1$ and $v_2$, we simply write $e = \{v_1,v_2\}$. It is called admissible, if $v_{1,n+1}\neq v_{2,n+1}$. If $v_{1,n+1} < v_{2,n+1}$, we call $v_1$ and $v_2$ the minor and major vertex of~$e$, respectively, and denote them by~$\mathrm{m}(e)$ and $\mathrm{M}(e)$. Let $P_e$ be the projection on $\mathbb{R}^{n+1}$ that projects on~$\mathbb{R}^n\times \{0\}$ along lines parallel to $e$. The barrier cone of~$e$, denoted by $C(e)$, is the smallest line-free convex cone that contains $P_e(\mathrm{supp}(p)) - P_e(e)$. It is identified with its projection on the first $n$ coordinates in $\mathbb{R}^n$. Its dual cone is
\begin{equation*}
C(e)^* := \{ w\in \mathbb{R}^n : \langle w,v \rangle \leq 0 \text{ for all } v \in C(e) \}.
\end{equation*}
A vector $w\in\mathbb{R}^n$ defining a total order on $\mathbb{Q}^n$ is said to be compatible with $e$ when $w\in C(e)^*$.
%, i.e. when~$C(e)$ is compatible with the order induced by $w$. 
Given an admissible edge $e = \{v_1,v_2\}$, we denote its slope with respect to its last coordinate by
\begin{equation*}
\mathrm{S}(e):= \frac{1}{v_{2,n+1}-v_{1,n+1}}(v_{2,1}-v_{1,1},\dots,v_{2,n}-v_{1,n}).
\end{equation*}
%and a compatible vector $w\in \mathbb{R}^n$ that induces a total order $\preceq$ on $\mathbb{Q}^n$, the Newton-Puiseux algorithm determines $|v_{1,n+1}-v_{2,n+1}|$ many series roots $\phi$ of $p(x,y)$ whose leading exponent $\mathrm{lexp}_\preceq(\phi)$ is $- S(e)$.\\



%
%
%Series expansions of multivariate algebraic functions were studied by MacDonald~\cite{..},
%who generalized the classical Newton-Puiseux algorithm~\cite{..,..,..} from the univariate
%case to the case of several variables. In this setting, we are interested in infinite series
%\[
%  a = \sum_{i_1,\dots,i_n\in\set Q} a_{i_1,\dots,i_n} x_1^{i_1}\cdots x_n^{i_n}
%\]
%with coefficients $a_{i_1,\dots,i_n}$ in some field~$K$.
%The \emph{support} of such a series is defined as the set of exponent vectors of all the
%terms which have a nonzero coefficient:
%\[
%  \supp(a) := \{ (i_1,\dots,i_n)\in\set Q : a_{i_1,\dots,i_n}\neq0 \}.
%\]
%Like in the univariate case, the exponents are allowed to be rational, but they should
%have a finite common denominator, so we restrict the attention to series $a$ for which
%there exists a $k\in\set Z$ with $k\supp(a)\in\set Z^n$.
%
%In order to be able to multiply series, we need to restrict their supports further. 
%A cone is a subset of $\set R^n$ which closed under multiplication by nonnegative real numbers.
%A cone is convex if it is also closed under addition.
%It is called strongly convex or line-free or pointed if it does not contain a line.
%It is called polyhedral if there are $v_1,\dots,v_n\in\set R^n$ such the elements of the
%cone are precisely the linear combinations of $v_1,\dots,v_n$ with nonnegative real coefficients.
%It is called rational polyhedral if the $v_1,\dots,v_n$ can be chosen from~$\set Q^n$.
%In this paper, we will only consider strongly convex rational polyhedral cones, and we will
%simply call them cones from now on.
%
%For a cone $C\subseteq\set R^n$ and a field~$K$, MacDonald defines $K_C[[x_1,\dots,x_n]]$ as the
%ring of all series $a$ such that $\supp(a)\subseteq C\cap\frac1k\set Z^n$ for some $k\in\set Z$.
%It turns out that $K_C[[x_1,\dots,x_n]]$ is indeed a ring (see also \cite{..}). It is however
%not a field. In order to get a field, we impose an order on the set of all terms $x_1^{e_1}\cdots x_n^{e_n}$.
%The order must be compatible with multiplication in the sense that we want
%$\sigma\leq\tau\Rightarrow\rho\sigma\leq\rho\tau$ to be true for all terms $\sigma,\rho,\tau$.
%A quick way to obtain such an order is to pick a vector $w=(w_1,\dots,w_n)\in\set R^n$
%whose coordinates are linearly independent over $\set Q$ and then to define
%\[
%x_1^{u_1}\cdots x_n^{u_n}
%\leq
%x_1^{v_1}\cdots x_n^{v_n}
%\iff
%\sum_{i=1}^n w_iu_i \leq \sum_{i=1}^n w_iv_i.
%\]
%Geometrically, the vector $w$ defines a halfspace $H=\{ z\in\set R^n : z\cdot w\geq0 \}\subseteq\set R^n$,
%and $\sigma\leq\tau$ means that there is a translated
%copy of $H$ which contains the exponent vector of $\tau$ but not the exponent vector of~$\sigma$.
%\begin{center}
%  \begin{tikzpicture}[scale=.2]
%    \fill[lightgray] (-4,6)--(6,6)--(6,-6)--(4,-6)--cycle;
%    \draw[->](-6.5,0)--(6.5,0);
%    \draw[->](0,-6.5)--(0,6.5);
%    \clip (-6,-6) rectangle (6,6);
%    \draw[thick] (-4,6)--(4,-6) (3,4.5) node {$H$};
%    \draw[thick,->] (0,0)--(1.8,1.2) node[right] {$w$};
%    \draw[xshift=-2.5cm] (-4,6)--(4,-6)
%       (-3,3) node {$\bullet$} node[below] {$\sigma$}
%       (0,1) node {$\bullet$} node[above] {$\tau$} ;
%  \end{tikzpicture}
%\end{center}
%We say that a cone $C$ is compatible with an order $\leq$ if $C$ has
%a minimal element w.r.t.~$\leq$. Geometrically, this means that $C$ is contained in the
%halfspace~$H$. If $\mathcal{C}$ denotes the set of all cones that are compatible with
%a fixed term order~$\leq$, then we set
%\[
%  K_\leq((x_1,\dots,x_n)) := \bigcup_{e\in\set Q^n}\bigcup_{C\in\mathcal{C}} x^e K_C[[x_1,\dots,x_n]].
%\]
%The elements of $K_\leq((x_1,\dots,x_n))$ are thus all the series whose support is contained
%in a shifted copy of a cone that is compatible with~$\leq$.
%It turns out that $K_\leq((x_1,\dots,x_n))$ is a field (cf.~\cite[Thm. ??]{..}).
%
%The main result of MacDonald (his Thm.~3.6, slightly rephrased) asserts the following:
%if $K$ is an algebraically closed field, $\leq$ is a term order, and $p\in
%K[x_1,\dots,x_n][y]$ is a squarefree polynomial of degree~$d>0$ in~$y$, then
%$K_\leq((x_1,\dots,x_n))$ contains $d$ distinct roots of~$p$. Moreover, there is
%an algorithm which for any given $N\in\set N$ can compute the $N$ smallest terms
%(w.r.t. $\leq$) of each of these roots. The algorithm also returns a cone which
%is compatible with $\leq$ and has the property that the exponent vectors of all
%the remaining terms belong to the shifted copy of this cone rooted at the exponent
%vector of the last computed term.
%\begin{center}
%  \begin{tikzpicture}[scale=.2]
%    \fill[lightgray] (-4,6)--(6,6)--(6,-6)--(4,-6)--cycle;
%    \fill[gray] (3,2)--(6,-1)--(6,6)--(3,6)--cycle;
%    \draw[->](-6.5,0)--(6.5,0);
%    \draw[->](0,-6.5)--(0,6.5);
%    \clip (-6,-6) rectangle (6,6);
%    \foreach \x/\y in {0/1, 0/2, 0/-1, 2/3, -1/2, -1/1, 1/1, 1/-2, 1/-3, 2/0, 3/0, 3/2} \draw (\x,\y) node {$\bullet$};
%  \end{tikzpicture}
%\end{center}
%
%// connection to recent mexican papers
%
%An element of $K_\leq((x_1,\dots,x_n))$ is called \emph{algebraic} if it is a root of
%a polynomial $p\in K[x_1,\dots,x_n][y]\setminus\{0\}$, otherwise transcendental. If
%$a\in K_\leq((x_1,\dots,x_n))$ is algebraic, then for every $i\in\{1,\dots,n\}$ it
%also satisfies a linear differential equation of the form 
%\[
%p_0 a + p_1 \frac{\partial}{\partial x_i} a + \cdots + p_r \frac{\partial^r}{\partial x_i^r} a = 0
%\]
%with $p_0,\dots,p_r\in K[x_1,\dots,x_n]$ not all zero (Abel's theorem~\cite{..}).
%Conversely, not every series $a\in K_\leq((x_1,\dots,x_n))$ which satisfies such
%a system of $n$ linear differential equations is algebraic. A series satisfying
%such a system is called \emph{D-finite}~\cite{..,..,..}.
%
%D-finiteness is a robust concept in the sense that there are many operations which
%can be applied to a D-finite series without destroying its D-finiteness. For example,
%if $a\in K_\leq((x_1,\dots,x_n))$ is D-finite, then so is its \emph{positive part,}
%defined by
%\[
%  [x_1^>\cdots x_n^>]
%  \sum_{i_1,\dots,i_n\in\set Q} a_{i_1,\dots,i_n} x_1^{i_1}\cdots x_n^{i_n}
%  := \sum_{i_1,\dots,i_n>0} a_{i_1,\dots,i_n} x_1^{i_1}\cdots x_n^{i_n}.
%\]
%\begin{center}
%  \begin{tikzpicture}[scale=.2]
%    \fill[gray] (-2,-3)--(-5,6)--(4,6)--(-2,-3)--cycle;
%    \draw[->](-6.5,0)--(6.5,0);
%    \draw[->](0,-6.5)--(0,6.5);
%    \draw (10,0) node {$\to$};
%    \begin{scope}[xshift=20cm]
%      \begin{scope}
%        \clip (-2,-3)--(-5,6)--(4,6)--(-2,-3)--cycle;
%        \foreach \x in {-7,-6.6,...,2} \draw (\x,-3)--(\x+1,6);
%      \end{scope}
%      \begin{scope}
%        \clip (.2,0) rectangle (6,6);
%        \fill[gray] (-2,-3)--(-5,6)--(4,6)--(-2,-3)--cycle;
%      \end{scope}
%      \draw[->](-6.5,0)--(6.5,0);
%      \draw[->](0,-6.5)--(0,6.5);
%    \end{scope}
%  \end{tikzpicture}
%\end{center}
%This was first shown by Lipshitz~\cite{..} and later~\cite{..} updated to a more
%modern algorithm for computing a system of differential equations for
%$[x_1^>\cdots x_n^>]a$ from a given system of differential equations for~$a$.
%These papers do not discuss fractional exponents, but since our rational exponents
%are required to have a finite common denominator, this does not make any difference.
%Moreover, the constructions are not limited to positive part extraction but also
%allow us to compute for any given D-finite series $a\in K_\leq((x_1,\dots,x_n))$
%(specified by a system of differential equations) and any given cone
%$C\subseteq\set R^n$ (specified by a set of rational generators) a system of
%differential equations for the series
%\[
%  [C] \sum_{i_1,\dots,i_n\in\set Q} a_{i_1,\dots,i_n} x_1^{i_1}\cdots x_n^{i_n}
%  := \sum_{(i_1,\dots,i_n)\in C\cap\set Q^n} a_{i_1,\dots,i_n} x_1^{i_1}\cdots x_n^{i_n}.
%\]
%
%Another operation which preserves D-finiteness is the composition with
%algebraic functions. We need only the following special case of this:
%if $a=a(x_1,\dots,x_n)\in K_\leq((x_1,\dots,x_n))$ is D-finite and
%$\tau_1,\dots,\tau_n$ are terms whose exponent vectors are linearly
%independent, then $a(\tau_1,\dots,\tau_n)$ is a D-finite series
%(possibly no longer belonging to $K_\leq((x_1,\dots,x_n))$ but to
%a field with some other term order). This
%feature allows us to turn arbitrary cones into more comfortable positions.
%\begin{center}
%  \begin{tikzpicture}[scale=.2]
%    \fill[gray] (2,3.5)--(-6,5.5)--(-6,-6)--(6,-6)--(6,-4)--cycle;
%    \draw[->](-6.5,0)--(6.5,0);
%    \draw[->](0,-6.5)--(0,6.5);
%    \draw (10,0) node {$\to$};
%    \begin{scope}[xshift=20cm]
%      \fill[gray] (-1,-2)--(-3,6)--(4,6)--cycle;
%      \draw[->](-6.5,0)--(6.5,0);
%      \draw[->](0,-6.5)--(0,6.5);
%    \end{scope}
%  \end{tikzpicture}
%\end{center}
%Note in particular that the support of an element $a\in K_\leq((x_1,\dots,x_n))$ can in general
%involve for each variable $x_i$ arbitrarily high as well as arbitrarily small exponents.
%With the above substitution, we can in particular always turn an obtuse cone into an acute cone.
%This is useful because such series can be viewed as univariate
%Puiseux series in one variable, say~$x_n$, with coefficients that are Puiseux polynomials
%in the remaining variables $x_1,\dots,x_{n-1}$. This in turn is useful because univariate
%D-finite series are somewhat easier to handle than multivariate ones. In particular, writing
%$a=\sum_{m\in\set Z} a_m(x_1,\dots,x_{n-1})x_n^{m/k}$, the differential equation for $a$
%w.r.t.\ $x_n$ can be translated into a linear recurrence equation
%\[
%  p_0 a_m + \cdots + p_s a_{m+s} = 0
%\]
%with coefficients $p_0,\dots,p_s\in K[x_1,\dots,x_{n-1},m]$, $p_s\neq0$, for the
%coefficients of~$a$ (cf. \cite{..}). Viewed as elements of $K(x_1,\dots,x_{n-1})[m]$, the polynomial
%$p_s$ can only have finitely many roots. This has important consequences.
%For example, it is impossible for the coefficient sequence $(a_m)_{m\in\set Z}$ to have
%arbitrarily large gaps.
%\begin{center}
%  \begin{tikzpicture}[scale=.2]
%    \fill[gray] (-4,-4.5)--(-3,6)--(5,6)--cycle;
%    \draw[->](-6.5,0)--(6.5,0);
%    \draw[->](0,-6.5)--(0,6.5);
%    \draw (-6,1)--(6,1) (-6,3)--(6,3);
%    \draw (6,2) node[right] {\rlap{$\leftarrow$ maximum gap size}};
%    \clip (-6,1) rectangle (6,3);
%    \fill[lightgray] (-4,-4.5)--(-3,6)--(5,6)--cycle;
%    \foreach\x in {-6.8,-6.5,...,6} \draw (\x,1)--(\x+.2,3);
%  \end{tikzpicture}
%\end{center}
%This observation was recently used in~\cite{..} in order to
%prove that certain elements of $K_\leq((x_1,\dots,x_n))$ cannot be D-finite (and
%hence in particular not be algebraic).
%Another consequence is that there is some finite and computable $s_{\max}\in\set Z$
%such that $a_m=0$ for all $m\leq s_{\max}$ implies $a_m=0$ for all $m\in\set Z$.
%We will exploit this feature in our equality test. 



%We collect some definitions before we present the algorithm.
%
%\begin{Definition}
%Let $p(x) = \sum_I a_I x^I$ be a Puiseux polynomial in $n+1$ variables. Its Newton polytope $\mathrm{NP}(p)$ is the convex hull $\mathrm{conv} \{I \in\mathbb{Q}^{n+1} \mid a_I \neq 0 \}$ in $\mathbb{R}^{n+1}$. Given an edge $e$ between vertices $r = (r_1,\dots,r_{n+1})$ and $s=(s_1,\dots,s_{n+1})$, we say that it is admissible if $r_{n+1}\neq s_{n+1}$. In this case we denote its slope with respect to $x_{n+1}$ by 
%\begin{equation*}
%S(e) = \frac{1}{s_{n+1}-r_{n+1}}(s_1-r_1,\dots,s_{n}-r_{n}).
%\end{equation*}
%If $r_{n+1} < s_{n+1}$ we write $\mathrm{m}(e) = r$ and $\mathrm{M}(e) = s$ and call $r$ and $s$ the minor and major vertex of $e$, respectively.
%Let $L$ be the line which contains $r$ and $s$. The barrier wedge $W(e)$ of the Newton polytope of $p$ with respect to $e$ is defined by
%\begin{equation*}
%W(e) = \{ \lambda (r-t) + t \mid \lambda \in\mathbb{R}_{\geq 0},\ r\in\mathrm{NP}(p),\ t\in L \}.
%\end{equation*}
%If $e$ is admissible, its intersection with $\{x\in\mathbb{R}^{n+1} \mid x_{n+1}=0\}$ is of the form $y + C(e)$ for some $y\in\mathbb{R}^{n+1}$, and some strictly convex polyhedral cone $C(e)$. We refer to $C(e)$ as the barrier cone of $e$ and identify it with its projection on the first $n$ coordinates in $\mathbb{R}^n$.
%\end{Definition}

%\subsection{The Newton-Puiseux algorithm}

We now present the Newton-Puiseux algorithm. For details, in particular for a proof of its correctness, we refer to~\cite[Theorem~3.5]{MacDonald}.

\begin{Algorithm}[Newton-Puiseux Algorithm]\label{alg:NPA}
Input: A square-free and non-constant polynomial $p\in\mathbb{K}[\bold{x},y]$, an admissible edge $e$ of its Newton polytope, an element $w$ of the dual of its barrier cone $C(e)$ defining a total order on $\mathbb{Q}^n$, and an integer $k$.\\
  Output: A list of $\mathrm{M}(e)_{n+1}-\mathrm{m}(e)_{n+1}$ many pairs $(c_1\bold{x}^{\alpha_1}+\dots+c_N \bold{x}^{\alpha_N},C)$ with $c_1\bold{x}^{\alpha_1},\dots,c_N\bold{x}^{\alpha_N}$ being the first $N$ terms of a series solution $\phi$ of $p(\bold{x},\phi) = 0$, ordered with respect to $w$, and $C$ being a line-free cone such that $\mathrm{supp}(\phi)\subseteq \{\alpha_1,\dots,\alpha_{N-1}\} \cup \left(\alpha_N + C\right)$, where $N\geq k$ is minimal such that the series solutions can be distinguished by their first $N$ terms.
  \step 10 Compute the roots $c$ of $p_e(t)=\sum_{I} a_I t^{I_{n+1}-\mathrm{m}(e)_{n+1}}$, where $a_I = [(\bold{x},y)^I]p$ and the sum runs over all $I$ in~$e\cap \mathrm{supp}(p)$, set $L$ equal to the list of pairs~$(\phi,e)$ with $\phi = c x^{-\mathrm{S}(e)}$ and $N$ equal to~$1$.
  \step 20 While $|L|\neq \mathrm{M}(e)_{n+1}-\mathrm{m}(e)_{n+1}$ or $N< k$, do:
  \step 31 Set $\tilde{L} = \{\}$ and $N = N+1$.
  \step 41 For each $(\phi,e)\in L$ with $\phi$ not having $k$ terms or $p_e(t)$ not having only simple roots, do:
  \step 52 If $\phi$ satisfies $p(\bold{x},\phi)=0$, append $(\phi,e)$ to $\tilde{L}$, otherwise compute the Newton polytope of $p(\bold{x},\phi+y)$ and determine its unique edge path $e_1,\dots,e_k$ such that $\mathrm{m}(e_1)_{n+1}$ equals zero, and $\mathrm{M}(e_k)$, but not $\mathrm{m}(e_k)$, lies on the line through $e$, and $w\in \bigcap C^*(e_i)$.
  \step 62 For each edge $e$ of the edge path, do:
  \step 73 Compute the roots $c$ of $p_e(t) = \sum_{I} a_I t^{I_{n+1}-\mathrm{m}(e)_{n+1}}$, where $a_I = [(\bold{x},y)^I]p(\bold{x},\phi+y)$ and the sum runs over all elements $I$ in $e\cap \mathrm{supp}(p(\bold{x},\phi+y))$, and append to $\tilde{L}$ all pairs $(\phi + c\bold{x}^{-\mathrm{S}(e)},e)$.
  \step 81 Set $L = \tilde{L}$.
  \step 90 Replace each pair $(\phi + c\bold{x}^{-\mathrm{S}(e)},e)$ of $L$ by $(\phi + c\bold{x}^{-S(e)}, C)$, where $C$ is the barrier cone of $e$ with respect to $p(\bold{x},\phi+y)$, and return $L$. 
\end{Algorithm}

\section{Finite encodings of algebraic series}

The Newton-Puiseux algorithm allows to determine the series solutions of a polynomial equation term by term. The next proposition implies that it can also be used to represent a series by a finite amount of of data: its minimal polynomial, a total order, and the first few terms of it with respect to this order.

\begin{Proposition}\label{prop:uniqueness}
Let $p\in\mathbb{K}[\bold{x},y]$ be square-free and non-constant, $e$ an admissible edge of its Newton polytope, $w\in C^*(e)$ defining a total order on $\mathbb{Q}^n$ and $a_1\bold{x}^{\alpha_1},\dots,a_N \bold{x}^{\alpha_N}$ the first few terms of a series solution $\phi$ as output by Algorithm~\ref{alg:NPA} when applied to $p,e,w$ and~$k=0$. Then $\phi$ is the only series solution whose first terms with respect to the total order defined by $w$ are $a_1\bold{x}^{\alpha_1},\dots,a_N \bold{x}^{\alpha_N}$.
\end{Proposition}
\begin{proof}
By design of the algorithm, the first $N$ terms of any other series solution constructed from $e$ differ from $a_1\bold{x}^{\alpha_1},\dots,a_N\bold{x}^{\alpha_N}$. If $\tilde{e}$ is another edge such that~$w\in C^*(\tilde{e})$, then the leading exponent of any series solution resulting from it is $-\mathrm{S}(\tilde{e})$ and different from the leading exponent $-\mathrm{S}(e)$ of $\phi$. Since all series solutions which have a leading exponent with respect to $w$ are constructed from such edges, this finishes the proof.
%. Since $S(\tilde{e})$ does not equal $S(e)$, the corresponding series do not equal~$\phi$. 
%then the leading exponent of any series solution resulting from~$\tilde{e}$ and $w$ equals $-S(\tilde{e})$, i.e. the negative of the slope of $\tilde{e}$ with respect to the last coordinate. To finish the proof it is sufficient to note that $S(\tilde{e})\neq S(e)$.
\end{proof}

We illustrate the Newton-Puiseux algorithm and Proposition~\ref{prop:uniqueness} with a first example.

\begin{Example}\label{ex:NPA}
We determine the first terms of a series solution of the equation
\begin{equation*}
p(x,y,z) := 4x^2y+(x^2y+xy^2+xy+y)^2-z^2 = 0
\end{equation*} 
The Newton polytope of $p$ has four admissible edges, one of which is the edge~$e=\{(0,2,0),(0,0,2)\}$. Its barrier cone is $C(e)=\langle (1,1),(2,-1) \rangle$, and $w:=(-\sqrt{2},-1)$ is an element of its dual $C^*(e)$. Its components are linearly independent over $\mathbb{Q}$, therefore it defines a total order $\preceq$ on $\mathbb{Q}^2$. By~\cite[Theorem 3.5]{MacDonald}, and because the projection of $e$ on its last coordinate has length $2$, there are two series solutions~$\phi_1$ and $\phi_2$ of~$p(x,y,z)=0$ in $\mathbb{C}_\preceq((x,y))$. We determine their first terms using Algorithm~\ref{alg:NPA}. The slope of $e$ is~$\mathrm{S}(e) = (0,-1)$, so the solutions have a term of the form $c y$, for some $c\in\mathbb{C}$. The coefficients $c$ are the solutions to $-1+t^2 = 0$. Hence, $y$ is the first term of one series solution, say $\phi_1$, and $-y$ the first term of~$\phi_2$. Furthermore, their support is contained in $(0,1) + \langle (1,1),(2,-1)\rangle$. To compute the next term of $\phi_1$, for instance, we consider the polynomial $p(x,y,y+z)$. The edge path on its Newton polytope mentioned in Algorithm~\ref{alg:NPA} consists of the single edge $e=\{(1,2,0),(0,1,1)\}$. Its slope with respect to the last coordinate is~$(-1,-1)$, so its next term is of the form $c xy$, where $c$ is the root of $-2+2t$. By Proposition~\ref{ex:NPA} the series $\phi_1$ and $\phi_2$ can be encoded by $(p,(-\sqrt{2},-1),y)$ and $(p,(-\sqrt{2},-1),-y)$, respectively.
\end{Example}

\section{An effective equality test for algebraic series}\label{sec:equ}

The encoding of an algebraic series by its minimal polynomial, a total order, and its first terms is not unique, and so it is natural to ask if it is possible to decide whether two encodings represent the same series. We clarify this now. Assume that~$\phi_1$ and $\phi_2$ are two series solutions of $p(\bold{x},y)=0$ encoded by~$(p,w_1,p_1)$ and $(p,w_2,p_2)$ where $w_1$ and~$w_2$ are elements of $\mathbb{R}^n$ inducing total orders~$\preceq_1$ and~$\preceq_2$ on $\mathbb{Q}^n$ and $p_1$ and $p_2$ are Puiseux polynomials in $\bold{x}$ representing the sum of the first terms of $\phi_1$ and $\phi_2$ with respect to $\preceq_1$ and $\preceq_2$, respectively. Using the Newton-Puiseux algorithm we can assume that the trailing term of $p_2$ with respect to $\preceq_2$ is smaller than the trailing term of $p_1$ with respect to $\preceq_2$. If the sequence of terms of $p_1$ does not agree with the initial sequence of terms of $p_2$ when ordered with respect to $\preceq_1$, then this proves that $\phi_1$ and $\phi_2$ are not the same. The next example demonstrates that we can also prove equality of series by comparing finitely many of their initial terms and determining an estimate of their supports.

\begin{Example}\label{ex:nonUniqueness}
The Newton polytope of 
\begin{equation*}
p(x,y,z):= x+y - (1+x+y) z
\end{equation*}
has four admissible edges from each of which we can compute the first terms of a series solution of~$p(x,y,z) = 0$. These series can be encoded by 
\begin{align*}
(p, (-1+1/\sqrt{2},1),1) \quad  &\text{and} \quad (p, (-1+1/\sqrt{2},-1),1), \quad \text{and}\\ 
(p, (-1+1/\sqrt{2},-2),x) \quad &\text{and} \quad (p, (-2+1/\sqrt{2},-1),y).
\end{align*}
%In Example~\ref{ex:inv} we saw that
Since there are only three series which qualify as multiplicative inverses of $1+x+y$, two of the above encodings have to represent the same series. We claim that the series~$\phi_1$ represented by $(p, (-1+1/\sqrt{2},-2),x)$ and~$\phi_2$ represented by $(p, (-2+1/\sqrt{2},-1),y)$ are equal. The order of $y$ with respect to~$(-1+1/\sqrt{2},-2)$ is $-2$, and the terms of $\phi_1$, whose order with respect to~$(-1+1/\sqrt{2},-2)$ is at least~$-2$, are
\begin{equation*}
x,-x^2,x^3-x^4,x^5-x^6,y.
\end{equation*}
Ordering these terms with respect to $(-2+1/\sqrt{2},-1)$ results in the sequence
\begin{equation*}
y,x,-x^2,x^3,-x^4,x^5,-x^6,
\end{equation*}
whose first term equals the first term of $\phi_2$. Algorithm~\ref{alg:NPA} shows that the support of $\phi_1$ is contained in a shift of $\langle (0,1),(7,-1) \rangle$ and that the support of $\phi_2$ is contained in a shift of $\langle(0,1),(1,-1) \rangle$. These cones, $\langle (0,1),(7,-1) \rangle$ and $\langle(0,1),(1,-1) \rangle$, are compatible in the sense that their sum is a line-free cone. Consequently, there is a total order $\preceq$ on $\mathbb{Q}^2$ that is compatible with both of them, hence~$\mathbb{C}_{\preceq}((x,y))$ contains $\phi_1$ as well as $\phi_2$. Since, by construction, $\phi_1$ and $\phi_2$ are roots of $p$, and since $p$ has degree~$1$ with respect to $z$ and $\mathbb{C}_{\preceq}((x,y))$ is a field and therefore can only contain at most one solution of $p(x,y,z) = 0$, the series $\phi_1$ and $\phi_2$ have to be the same. 
\end{Example}

To see that the argument just given applies in more generality, we recall that $\mathbb{K}_{\preceq_1}((\bold{x}))$ contains a complete set of series solutions of $p(\bold{x},y)=0$. Assume that $\phi_1$ is constructed from an edge $e$ of the Newton polytope of $p$ for which $w_1\in C^*(e)$. We claim that $e$ can be extended to a unique maximal edge path $\{e_1,\dots,e_k\}$ for which $w\in \bigcap C^*(e_i)$ so that the series resulting from these edges form a complete set of series solutions of $p(\bold{x},y)=0$ in $\mathbb{K}_{\preceq_1}((\bold{x}))$. Starting with the edge~$e$, let $e'$ and $e''$ be those edges of the Newton polytope of $p(\bold{x},y)$ with $\mathrm{M}(e') = \mathrm{m}(e)$ and $\mathrm{m}(e'') = \mathrm{M}(e)$ for which each of~$\langle w_1, \mathrm{m}(e') \rangle$ and~$\langle w_1, \mathrm{M}(e'') \rangle$ is maximal. Extending $e'$ and $e''$ analogously as well as the subsequent edges until none of them can be extended results in a path with the claimed property. The uniqueness of this path is immediate: if it were not unique then $\mathbb{K}_{\preceq_1}((\bold{x}))$ would contain more than $\deg_y(p)$ many series solutions, contradicting $\mathbb{K}_{\preceq_1}((\bold{x}))$ being a field. Assume that the sequences of initial terms of $\phi_1$ and $\phi_2$ are the same when ordered with respect to $\preceq_1$, and let $C_2$ be the cone output by Algorithm~\ref{alg:NPA} such that $\mathrm{supp}(\phi_2)\subseteq\{\beta_1,\dots,\beta_{M-1}\}\cup \left( \beta_M + C_2 \right)$ where $\beta_M$ is the trailing exponent of~$p_1$ with respect to $w_2$. If~$w_1\in C_2^*$ then $\phi_2$ is an element of $\mathbb{K}_{\preceq_1}((\bold{x}))$ and it follows that $\phi_2=\phi_1$ as in Example~\ref{ex:nonUniqueness}. If we assume that $C_2$ is the minimal cone for which~$\mathrm{supp}(\phi_2)\subseteq\{\beta_1,\dots,\beta_{M-1}\}\cup \left( \beta_M + C_2 \right)$ then we can also conclude that~$\phi_2\neq \phi_1$ if $w_1\notin C_2^*$. The latter holds because $C_2$ being minimal and~$w_1$ not being an element of~$C_2^*$ implies that there is an element of $\mathrm{supp}(\phi_2)$ in $\beta_M + C_2$ that is larger than $\beta_M$ with respect to $w_1$. But if~$\phi_1=\phi_2$, the set of elements of $\mathrm{supp}(\phi_2)$ not smaller than $\beta_M$ is $\mathrm{supp}(p_1)$, and by construction $\mathrm{supp}(p_1)$ is a subset of $\{\beta_1,\dots,\beta_M\}$. \\


%Let $\phi_1$ and $\phi_2$ be series solutions of $m(X)=0$ encoded by $(m(X),w_1,p_1)$ and~$(m(X),w_2,p_2)$ where $w_1$ and $w_2$ are elements of $\mathbb{R}^2$ inducing total orders~$\preceq_1$ and~$\preceq_2$ on $\mathbb{Q}^2$ and $p_1$ and $p_2$ are Laurent polynomials in $x$ and $y$ representing the sum of the first terms of $\phi_1$ and $\phi_2$ with respect to $\preceq_1$ and $\preceq_2$ respectively. 

%To see that the argument just given holds in general,%The reason is that if $\phi_1$ is constructed from an edge $e$ of the Newton polytope of $m(X)$ then there is a maximal set $\{e_1,\dots,e_n\}$ of admissible edges such that $\mathrm{M}(e_i) = \mathrm{m}(e_{i+1})$ for~$i=1,\dots,n-1$ and~$e_i = e$ for some~$i\in\{1,\dots,n\}$ and the edges can be chosen such that $\bigcap C^*(e_i)\neq \{0\}$. If $w\in \bigcap C^*(e_i)$ induces a total order $\preceq$ on $\mathbb{Q}^2$ then $\mathbb{C}_{\preceq}((x,y))$ contains a complete set of series solutions. 
 
%To check equality of $\phi_1$ and $\phi_2$ one can proceed as in Example~\ref{ex:nonUniqueness} by computing the terms of $\phi_2$ whose order with respect to $w_2$ is not smaller than the minimum of the orders of the terms of $p_1$ with respect to $w_2$. If the sequence of these terms does not equal the one of $p_1$ when ordered with respect to $w_1$ then $\phi_1$ and $\phi_2$ are not equal. 

%Example showing that two series are not the same by observing that their first terms are the same, but their supports are not compatible. 
%We illustrate how to decide that two series are not the same by observing that their supports are not compatible.

% \begin{Example}
% Let 
% \begin{equation*}
%p(x,y) := -x^3  - (1-x-x^3+x^4)y + (1-2x+x^2)y^2,
% \end{equation*}
% and note that it is the numerator of $(y-\frac{1}{1-x})(y-\frac{x^2}{1-\frac{1}{x}})$. Its Newton polytope has two admissible edges, namely $\{(0,1),(0,2)\}$ and $\{(4,1),(2,2)\}$, which give rise to two series $\phi_1$ and $\phi_2$ that can be encoded by 
% \begin{equation*}
% (p, -\sqrt{2},1) \quad \text{and} \quad (p,\sqrt{2},x^2).
% \end{equation*}
%The order of the first term of $\phi_1$ with respect to $\sqrt{2}$ is $0$ and the terms of $\phi_2$ whose order with respect to~$w_2$ is at least $0$ are $x^2, x$ and $1$, whose highest order term with respect to $-\sqrt{2}$ is $1$, being equal to the corresponding highest order term of $\phi_1$. Although the first terms of $\phi_1$ and $\phi_2$ are the same, $\phi_1$ and $\phi_2$ are not as 
%\begin{equation*}
%\mathrm{supp}(\phi_1)\subseteq \langle 1 \rangle \quad \text{and} \quad \mathrm{supp}(\phi_2)\subseteq \{2,1\}\cup \langle -1 \rangle,
%\end{equation*}
%and $\langle 1 \rangle$ and $\langle -1 \rangle$ are the minimal cones having this property.
% \end{Example}
 
The above reasoning relied on the cones output by Algorithm~\ref{alg:NPA} not being too big. Although they are not always minimal, we believe that they are always in certain situations. We will have more to say about this in Section~\ref{sub:support}, but for the moment we just state the following conjecture.
 
\begin{Conjecture}\label{conj:minimality}
If the polynomial $p\in\mathbb{K}[\bold{x}][y]$ is primitive and the integer $k\in\mathbb{N}$ input to Algorithm~\ref{alg:NPA} is large enough, then the cones $C$ output by it are minimal.
\end{Conjecture}

\begin{Theorem}\label{thm:equality}
The equality of two multivariate algebraic series can be decided effectively.
\end{Theorem}
If Conjecture~\ref{conj:minimality} were correct, then Theorem~\ref{thm:equality} were just a corollary of it. In the following, however, we give a proof of Theorem~\ref{thm:equality} that is independent of Conjecture~\ref{conj:minimality}, based on properties of D-finite functions.
\begin{proof}
We complete the equality test for the case that $w_1\notin C_2^*$ and $C_2$ is not minimal. Algorithm~\ref{alg:NPA} provides a cone $C_1$ such that $w_1\in C_1^*$ and $\mathrm{supp}(\phi_1)\subseteq \{\alpha_1,\dots,\alpha_{N-1}\}\cup (\alpha_N+C_1)$, where $\alpha_1, \dots,\alpha_N$ are the exponents of the first few terms of $\phi_1$ with respect to $w_1$. W.l.o.g. we assume that $\alpha_N = \beta_M$. If $\mathrm{supp}(\phi_2)\setminus\{\beta_1,\dots,\beta_{M-1}\}\subseteq \beta_M + C_1$, then $\phi_2\in\mathbb{C}_{\preceq_1}((x))$, and therefore $\phi_2 = \phi_1$. If $\mathrm{supp}(\phi_2)\setminus\{\beta_1,\dots,\beta_{M-1}\}\nsubseteq \beta_M + C_1$, then $\phi_2 \neq \phi_1$. It therefore remains to explain how to decide whether the support of an algebraic series $\phi$ is contained in a cone $C$. 
We do so by using basic properties of D-finite series. To simplify the argument we assume that $\mathrm{supp}(\phi)\subseteq \mathbb{Z}^n$. Let $\omega=(\omega_1,\dots,\omega_n)\in\mathbb{Z}^n$ be such that for each $i\in\mathbb{Z}$ there are only finitely many $\alpha\in\mathrm{supp}(\phi)$ for which $\langle \alpha, \omega\rangle = i$ and none if $i< 0$, and consider the series 
\begin{equation*}
\tilde{\phi} (\bold{x},t) := \phi (x_1 t^{\omega_1},\dots, x_n t^{\omega_n} ) %= \sum_{i\geq 0} \tilde{\phi}_i(x) t^i,
\end{equation*}
%which we view as a series in $t$ whose coefficients are Laurent polynomials in $x$. 
Since $\phi$ is algebraic, so is~$\tilde{\phi}$, and because every algebraic series is D-finite, so is $\tilde{\phi}$. 
%In particular, its coefficient sequence $(\tilde{\phi}_i)$ satisfies a linear recurrence 
%\begin{equation*}
%q_0(i)\tilde{\phi}_i + q_1(i)\tilde{\phi}_{i+1} + \dots + q_r(i)\tilde{\phi}_{i+r} = 0
%\end{equation*}
%with $q_0,\dots,q_r \in \mathbb{C}[x][n]$. 
Let
\begin{equation*}
[\phi]_C(\bold{x}) := \sum_{I\in C\cap \mathbb{Z}^n} \left([\bold{x}^I] \phi\right) \bold{x}^I
\end{equation*}
be the restriction of $\phi$ to $C$ and let $[\tilde{\phi}]_C(\bold{x},t) := [\phi]_C(x_1t^{\omega_1},\dots,x_n t^{\omega_n})$ be the restriction of~$\tilde{\phi}$ to $C\times \mathbb{R}_{\geq 0}$, by abuse of notation.
By closure properties of D-finite functions $[\tilde{\phi}]_C$ is D-finite, and so is the difference~$\tilde{\phi} - [\tilde{\phi}]_C$. In particular, when viewed as a series in $t$, the coefficients of $\tilde{\phi} - [\tilde{\phi}]_C$ satisfiy a linear recurrence relation of the form 
\begin{equation*}
q_0(k)c_k + q_1(k)c_{k+1} + \dots + q_r(k)c_{k+r} = 0
\end{equation*}
with $q_0,\dots,q_r \in \mathbb{K}[\bold{x}][k]$. Verifying whether $\mathrm{supp}(\phi) \subseteq C$ amounts to checking if $\tilde{\phi} - [\tilde{\phi}]_C = 0$ which can be done by comparing finitely many initial terms of $\tilde{\phi} - [\tilde{\phi}]_C$ to zero. 
%Let $C = \langle v_1 ,v_2 \rangle$. Then $[\tilde{\phi}]_C$ can be written as the iterated diagonal
%\begin{equation*}
%\mathrm{Diag}_{x_1,x_2} \mathrm{Diag}_{y_1,y_2} \frac{1}{(1-x_2^{v_{11}}y_2^{v_{12}})(1-x_2^{v_{21}}y_2^{v_{22}})} \tilde{\phi}(x_1,y_1,t),
%\end{equation*}
%consequently $[\tilde{\phi}]_C$ is D-finite, and it was explained in~\cite{hypergeometric} how to use creative telescoping to derive a recurrence relation for its coefficient sequence. Having recurrence relations for $\tilde{\phi}$ and $[\tilde{\phi}]_C$ we can use closure properties to derive a 
%recurrence relation for their difference $\tilde{\phi} - [\tilde{\phi}]_C$. Verifying whether $\mathrm{supp}(\phi) \subseteq C$ amounts to checking if $\tilde{\phi} - [\tilde{\phi}]_C = 0$ which can be done by comparing finitely many of its initial terms to zero. 
\end{proof}

\begin{Remark}
The equality test for algebraic series is effective because the closure properties of D-finite functions it is based on can be performed effectively. However, as was explained in~\cite{bostan2017hypergeometric}, it can be computationally quite expensive to do so.
\end{Remark}
%\begin{Proposition}
%Given a total order $\preceq$ on $\mathbb{Q}^2$ induced by an irrational vector $w\in\mathbb{R}^2$, the field $\mathbb{C}_{\preceq}((x,y))$ is algebraically closed. 
%\end{Proposition}

%\begin{Proposition}
%Given a polynomial $m(X)$ of $\mathbb{C}(x,y)[X]$ there is a bijective correspondence between the family of complete sets of series solutions of $m(X)=0$ contained in a field $\mathbb{C}_{\preceq}((x,y))$ with $\preceq$ is induced by an irrational vector of $\mathbb{R}^2$, and the sets of maximal edge paths $\{e_i\}$ of its Newton polytope for which~$\bigcap C^*(e_i)$ contains an irrational vector.
%\end{Proposition}

\section{The support of an algebraic series}\label{sub:support}
In Example~\ref{ex:nonUniqueness} we saw that the number of admissible edges of the Newton polytope of a polynomial equation is not necessarily bounded by the number of series solutions the equation has. It can happen that different edges give rise to the same series solution. For the purpose of encoding a series any edge is as good as any other edge as long as it gives rise to the same series. However, when interested in information about the support of a series solution, it is advisable to inspect all the edges which give rise to this series solution. We explain how the effective equality test for algebraic series allows to derive information about the convex hull of the support of an algebraic series.

\begin{Example}\label{ex:cones}\label{ex:support}
In Example~\ref{ex:nonUniqueness} we saw that the Newton polytope of 
\begin{equation*}
p(x,y,z) = x+y - (1+x+y) z
\end{equation*}
has two admissible edges,
\begin{equation*}
e_1 = \{(0,0,1),(1,0,0)\} \quad \text{and} \quad e_2 = \{(0,0,1),(0,1,0)\},
\end{equation*}
that give rise to two encodings, 
\begin{equation*}
(p, (-1+1/\sqrt{2},-2),x) \quad \text{and} \quad (p, (-2+1/\sqrt{2},-1),y), 
\end{equation*}
of one and the same series solution $\phi$ of $p(x,y,z) = 0$. The barrier cones of these edges are 
\begin{equation*}
C(e_1) = \langle (1,0),(-1,1)\rangle \quad \text{and} \quad  C(e_2) = \langle (0,1),(1,-1) \rangle,
\end{equation*}
so that, by Algorithm~\ref{alg:NPA}, we have 
\begin{equation*}
\mathrm{supp}(\phi) \subseteq (1,0) + C(e_1) \quad \text{as well as} \quad \mathrm{supp}(\phi) \subseteq (0,1) + C(e_2).
\end{equation*}
\end{Example}

%In Example~\ref{ex:nonUniqueness} we saw that a series solution $\phi$ of $m(X) = x+y - (1+x+y) X = 0$ is encoded by $(m(X), (-1+1/\sqrt{2},-2),x)$ and $(m(X), (-2+1/\sqrt{2},-1),y)$ and in Example~\ref{ex:cones} we pointed out that its support is contained in $(1,0) + \langle (1,0),(-1,1)\rangle$ and $(0,1) + \langle (0,1),(1,-1) \rangle$. We note that having only the latter information about the series represented by $(m(X), (-1+1/\sqrt{2},-2),x)$ and $(m(X), (-2+1/\sqrt{2},-1),y)$ it is by no means obvious that they are elements of the same field $\mathbb{C}_{\preceq}((x,y))$, let alone that they are the same, given that the sum of $\langle (1,0),(-1,1)\rangle$ and $\langle (0,1),(1,-1) \rangle$ is not a strictly convex cone. 
The next proposition shows that the convex hull of the support of an algebraic series has only finitely many vertices and indicates how they can be found. 

\begin{Proposition}\label{prop:support}
For any series root $\phi$ of a non-zero square-free polynomial~$p\in\mathbb{K}[\bold{x},y]$, there is a surjection from the set of edges of the Newton polytope of $p$ which give rise to the series solution $\phi$ to the set of vertices of the convex hull of its support. In particular, the convex hull of the support of an algebraic series is a polyhedral set.
\end{Proposition}
\begin{proof}
We claim that the function that maps an edge $e$ to its slope $-\mathrm{S}(e)$ has the required properties. If~$e$ is an edge that gives rise to the series solution $\phi$, then $-\mathrm{S}(e)$ is necessarily a vertex of its support as it is the maximal element with respect to a total order induced by some irrational vector, so the function is well-defined. The function is also surjective because for any vertex $\alpha$ of the convex hull of $\mathrm{supp}(\phi)$ there is some $w\in\mathbb{R}^n$ that induces a total order~$\preceq$ on~$\mathbb{Q}^n$ with respect to which $\alpha$ is the maximal element of $\mathrm{supp}(\phi)$. In particular, $\phi$ is an element of $\mathbb{K}_{\preceq}((\bold{x}))$. Since all series solutions of $p(\bold{x},y) = 0$ in~$\mathbb{K}_{\preceq}((\bold{x}))$ can be constructed by the Newton-Puiseux algorithm, there is also an edge $e$ from which $\phi$ can be constructed. 
%and there is only one such edge because $S(e) \neq S(\tilde{e})$ for any other edge $\tilde{e}$ giving rise to a series solution in $\mathbb{K}_{\preceq}((x))$.
 %Since $\alpha = -S(e)$, and since $w\notin C^*(\tilde{e})$ for any other edge~$\tilde{e}$ with~$\alpha = -S(\tilde{e})$, there is only one such edge. 
\end{proof}
The vertices of the convex hull of the support of $\phi$ can consequently be determined by identifying the admissible edges $e$ of the Newton polytope of~$p(\bold{x},y)$ that give rise to it. To determine also the bounded faces of the convex hull of the support of $\phi$ it is helpful to be able to compute, for each of its vertices~$v$, a line-free cone $C$ such that $\mathrm{supp}(\phi)\subseteq v + C$. For each such edge $e$ that gives rise to $\phi$ and for any~$w\in C^*(e)$ that defines a total order, Algorithm~\ref{alg:NPA} provides its first $N$ terms $a_1\bold{x}^{\alpha_1},\dots,a_N\bold{x}^{\alpha_N}$ with respect to $w$ and a line-free cone $C$ compatible with $w$ such that $\mathrm{supp}(\phi)\subseteq \{\alpha_1,\dots,\alpha_{N-1}\}\cup (\alpha_N + C)$. The cone $\tilde{C}$ generated by $C$ and~$\{\alpha_2-\alpha_1,\dots,\alpha_N-\alpha_1\}$ has the property that $\mathrm{supp}(\phi)\subseteq \alpha_1 + \tilde{C}$. It is also line-free since it is compatible with~$w$.

%Knowing the vertices of the convex hull of $\mathrm{supp}(\phi)$, and having for each vertex~$v$ a strictly convex but not necessarily minimal cone $C$ such that $\mathrm{supp}(\phi)\subseteq v + C$, we can recover the bounded faces of the convex hull of $\mathrm{supp}(\phi)$.
\begin{Example}
We continue Example~\ref{ex:support}. Proposition~\ref{prop:support} implies that the vertices of the convex hull of the support of $\phi$ are $v_1 = (1,0)$ and $v_2 = (0,1)$. Apart from the vertices $v_1$ and $v_2$ itself, the only possible bounded face is the convex hull of $v_1$ and $v_2$. Since 
\begin{equation*}
\mathrm{supp}(\phi)\subseteq v_1 + \langle (1,0), (-1,1) \rangle \quad \text{and} \quad \mathrm{supp}(\phi) \subseteq v_2 + \langle (0,1), (1,-1) \rangle,  
\end{equation*}
the line through $v_1$ and $v_2$ supports $\mathrm{conv}(\mathrm{supp}(\phi))$, so $\mathrm{conv}(\{v_1,v_2\})$ is a face of it.
\end{Example}

In general, the candidates for the bounded faces of the convex hull of $\mathrm{supp}(\phi)$ are the convex hulls of subsets of its set of vertices. Whether the convex hull of a subset $V$ of vertices is indeed a face of~$\mathrm{conv}(\mathrm{supp}(\phi))$ can be decided by computing for each $v\in V$ a line-free cone $C_v$ such that~$\mathrm{supp}(\phi)\subseteq v + C_v$ and checking if there is a hyperplane $H$ containing $\mathrm{conv}(V)$ whose complement is the union of two half-spaces one of which has bounded intersection with $\bigcap_{v\in V} (v+C_v)$ and does not contain any vertices of $\mathrm{conv}(\mathrm{supp}(\phi))$. Clearly, there is such a hyperplane if and only if $\mathrm{conv}(V)$ is a face.\\ 

We already pointed out that for deciding whether two algebraic series are equal or not, it is convenient that the cones output by Algorithm~\ref{alg:NPA} are not too big. The minimality of the cones is also important for getting a good estimate of the convex hull of the support of such a series. The following example, however, shows that they do not need to be minimal.
\begin{Example}
One of the two series solutions of 
 \begin{equation*}
  p(x,y,z) = (1-x)((1-y)z-1) = 0 
 \end{equation*}
is the geometric series 
\begin{equation*}
  \phi = 1 + y + y^2 + \dots  
\end{equation*}
Though the convex hull of its support is the cone generated by $(1,0)$, Algorithm~\ref{alg:NPA} only shows that 
\begin{equation*}
 \mathrm{supp}(\phi) \subseteq \langle (1,0), (0,1) \rangle.
\end{equation*}
The difference between the two cones is caused by the polynomial~$p\in\mathbb{C}[x,y][z]$ not being primitive: first getting rid of its content, and then applying Algorithm~\ref{alg:NPA} results in a cone that is minimal. 
\end{Example}
The non-primitivity is not the only possible reason for a cone output by Algorithm~\ref{alg:NPA} not being minimal.

\begin{Example}\label{ex:minimal0}
One of the solutions of 
\begin{equation*}
 p(x,y,z )= 1+x+y + (1 + xy +2y)z +yz^2 = 0
\end{equation*}
is
\begin{equation*}
 \frac{-1-2y-xy + \sqrt{1-2xy+4xy^2+x^2y^2}}{2y}.
\end{equation*}
It has a series expansion $\phi$ whose first terms with respect to $w = (-1+1/\sqrt{2},-1)$ are
\begin{equation*}
-1-x+xy+x^2y^2 + \dots. 
\end{equation*}
The closed form of $\phi$ together with Newton's generalized binomial theorem implies that the minimal cone containing $\mathrm{supp}(\phi)$ is $\langle (1,0), (1,2) \rangle$ though Algorithm~\ref{alg:NPA} only shows that it is contained in $\langle (1,0), (0,1) \rangle$. However, the algorithm also shows that
\begin{equation*}
 \mathrm{supp}(\phi) \subseteq \{(0,0)\} \cup \left( (1,0) + \langle (1,1), (0,1)\rangle \right),
\end{equation*}
where now $\langle (1,1), (0,1)\rangle$ is the minimal cone having this property, and computing another term and another cone, we find that
\begin{equation*}
 \mathrm{supp}(\phi) \subseteq \{(0,0), (1,0)\} \cup ((1,1) + \langle (1,1), (1,2) \rangle), 
\end{equation*}
where the cone $\langle (1,1), (1,2) \rangle$ is not only minimal but also has the property that~$(1,1)+\mathbb{R}_{\geq 0}\cdot (1,1)$ and $(1,1) + \mathbb{R}_{\geq 0}\cdot (1,2)$ contain infinitely many elements of~$\mathrm{supp}(\phi)$.
\end{Example}

For the series in the previous examples we could easily decide whether the corresponding cones given by the Newton-Puiseux algorithm were minimal or not, because the series were algebraic of degree~$1$ and~$2$, respectively, and therefore very explicit. It remains to clarify how this can be decided in general. We have just seen that for the first terms $a_1 \bold{x}^{\alpha_1},\dots, a_N \bold{x}^{\alpha_N}$ of an algebraic series $\phi$, the line-free cone~$C$ output by Algorithm~\ref{alg:NPA} for which $\mathrm{supp}(\phi) \subseteq \{\alpha_1,\dots,\alpha_{N-1}\}\cup (\alpha_N+C)$ is not necessarily minimal. However, if $C$ is minimal and if for each edge of $\alpha_N+C$ the exponent $\alpha_N$ is not the only element of $\mathrm{supp}(\phi)$ it contains, then its minimality can be verified simply by determining some of these elements.
%, or by investigating the Newton polytope of  $p(x,a_1 x^{\alpha_1} + \dots +  a_{N-1} x^{\alpha_{N-1}}+y)$ and observing that there have to be some at some point. 

\begin{Example}\label{ex:minimal1}
We continue Example~\ref{ex:minimal0}. The cone $\langle (1,1), (1,2) \rangle$ for which $\mathrm{supp}(\phi)\subseteq \{(0,0), (1,0)\}\cup ((1,1)+ \langle (1,1), (1,2) \rangle)$ is minimal because the first terms of $\phi$ with respect to $w = (-1+1/\sqrt{2},-1)$ are
\begin{equation*}
-1 - x + x y + x^2 y^2 - x^2 y^3 +\dots 
\end{equation*}
and $(2,2)\in(1,1)+\mathbb{R}_{>0} \cdot (1,1)$ and $(2,3) \in(1,1) + \mathbb{R}_{>0} \cdot (1,2)$. 
%Instead of explicitly computing these terms
%by Algorithm~\ref{alg:NPA} we could just observe that one of the terms of 
%\begin{equation*}
% m(-1-x+x y) = -x^2y^2+x^2y^3
%\end{equation*}
%lies on $(1,1)+\mathbb{R}_{>0} \cdot (1,1)$ and $(1,1) + \mathbb{R}_{>0} \cdot (1,2)$, respectively. 
\end{Example}

%\begin{Conjecture}
%When the input polynomial is primitive, then all the cones output by Algorithm~\ref{alg:NPA} are minimal from some point on.
%\end{Conjecture}

When the cone $C$ output by Algorithm~\ref{alg:NPA} is minimal but $\alpha_N$ is the only element of $\mathrm{supp}(\phi)$ that lies on an edge of $\alpha_N + C$ we are not able to show its minimality. Proving its minimality relates to the following open problem which closes this section.

\begin{Problem}\label{prob:faces}
Given an algebraic series in terms of its minimal polynomial, a total order and its first terms with respect to this order, determine the unbounded faces of the convex hull of its support.
\end{Problem}

%\begin{Example}
% The Newton polytope of $m(X) = 1+x+y + (1 + xy +2y)X +yX^2$ has seven admissible edges precisely two of which, 
% \begin{equation*}
%  e_1 = \{(0,0,0), (0,0,1)\} \quad \text{and} \quad e_2 = \{(1,0,0), (0,0,1)\},
% \end{equation*}
% give rise to encodings
% \begin{equation*}
%  (m(X),(-1+1/\sqrt{2},-1),-1) \quad \text{and} \quad (m(X),(1+1/\sqrt{2},-2),-x)
% \end{equation*}
% of the series $\phi$ we studied in Example~\ref{ex:minimal0} and Example~\ref{ex:minimal1}. Consequently, 
% the vertices of the convex hull of its support are~$v_1 = (0,0)$ and~$v_2 = (1,0)$, and by computing the barrier cones of $e_1$ and $e_2$ we find that 
% \begin{equation*}
%  \mathrm{supp}(\phi)\subseteq (0,0) + \langle (1,0), (0,1) \rangle \quad \text{and} \quad \mathrm{supp}(\phi) \subseteq (1,0) + \langle (-1,0), (1,1) \rangle. 
% \end{equation*}
% Together with 
% \begin{equation*}
%  \mathrm{supp}(\phi)\subseteq \{(0,0), (1,0)\} \cup ((1,1)+\langle (1,1), (1,2) \rangle 
% \end{equation*}
% and the minimality of $\langle (1,1), (1,2) \rangle$ shown before this implies that 
% \begin{equation*}
%  \mathrm{conv}(\mathrm{supp}(\phi)) = [0,1]\cdot (1,0) + \mathbb{R}_{\geq 0} \cdot (1,2) + \mathbb{R}_{\geq 0}\cdot (1,1).
% \end{equation*}
%\end{Example}


\section{Effective arithmetic for algebraic series}

The effectivity of the equality test for algebraic series implies the effectivity of their arithmetic: given encodings of two series $\phi_1$ and $\phi_2$, we discuss how to decide whether their sum $\phi_1+\phi_2$ and product $\phi_1\phi_2$ are well-defined / algebraic, and in case they are, how to determine finite encodings for them.\\

%explain how to decide whether that there is no total order $\preceq$ for which $\phi_1,\phi_2\in\mathbb{K}_{\preceq}((x))$.  and in case they are, how to derive a finite encoding for $\phi_1+\phi_2$ and $\phi_1\phi_2$.\\
%are algebraic, and in case they are, how to derive a finite encoding for them.\\

%In general, the sum and product of two algebraic series $\phi_1$ and $\phi_2$ do not need to be algebraic. We explain how their algebraicity can be decided and show how finite encodings for them can be derived from the ones of $\phi_1$ and $\phi_2$. 

%Given two algebraic series $\phi_1$ and $\phi_2$ in terms of~$(m_1(Z),w,p_1)$ and~$(m_2(Z),w,p_2)$ we explain how to determine finite encodings for $\phi_1 + \phi_2$ and $\phi_1\phi_2$. 
We first explain that in general the sum and product of two algebraic series do not need to be well-defined / algebraic. 

\begin{Example}
The polynomial 
\begin{equation*}
p(x,y) := (1-x) y -1 
\end{equation*}
has two series roots
\begin{equation*}
\phi_1 = 1 + x + x ^2 + \dots \quad \text{and} \quad \phi_2 = - x^{-1} - x^{-2} - x^{-3} - \dots
\end{equation*}
both of which are algebraic, by construction, but neither is their sum nor their product. Their product is not well-defined, since its coefficients involve infinite sums, and their sum is not algebraic, because its powers are not well-defined.
\end{Example}

The sum and product of two algebraic series $\phi_1$ and $\phi_2$ are algebraic if and only if there is a vector~$w\in\mathbb{R}^n$ that induces a total order $\preceq$ on $\mathbb{Q}^n$ with respect to which both~$\mathrm{supp}(\phi_1)$ and $\mathrm{supp}(\phi_2)$ have a minimal element. To decide whether such a vector exists, and in case it does, to find it, one would need to determine $\mathrm{conv}(\mathrm{supp}(\phi_1))$ and $\mathrm{conv}(\mathrm{supp}(\phi_2))$. However, we are only able to determine an estimate of the convex hull of the support of an algebraic series, see Section~\ref{sub:support}, in particular Problem~\ref{prob:faces}. Consequently, we can find such a vector only sometimes but not always when it exists, and of course, in case we cannot find one, it does not mean there is none.
Yet, to prove that the sum (and product) of $\phi_1$ and~$\phi_2$ is not algebraic, one could instead compute an annihilating polynomial~$p(\bold{x},y)$ for the potentially algebraic series~$\phi_1 + \phi_2$ and observe that none of its series roots equals the sum of $\phi_1$ and~$\phi_2$. 
%This can be done by comparing truncations of series roots of $m(X)$ with (the sum of) the initial terms of $\phi_1$ and $\phi_2$, or compare their . If $\phi_1 + \phi_2$ differs from all series roots of $m(X)$, then it cannot be algebraic. 
%and the initial terms of the remaining one equal the ones of~$\phi_1+\phi_2$, 
%Of course, if all but one series root of $m(X)$ have been identified as being different from~$\phi_1+\phi_2$, 
%this is not a proof that the series are equal, even though they might seem very likely to be. 

\begin{Example}
Let 
\begin{equation*}
p_1(x,y) := \left(1+x+x^2 -y)(x^2-(1-x)y\right) \quad \text{and} \quad p_2(x,y) := y \left(x^2-(x-1)y\right)
\end{equation*}
and consider the series 
\begin{equation*}
\phi_1 = x^2 + x^3 + x^4 + \dots \quad \text{and} \quad \phi_2 = x + 1 + x^{-1} \dots 
\end{equation*}
encoded by
\begin{equation*}
(p_1,-\sqrt{2},x^2) \quad \text{and} \quad (p_2,\sqrt{2},x).
\end{equation*}
We can show that $\phi_1 + \phi_2$ is not algebraic by computing the generator
\begin{equation*}
p(x,y) = \left(1 + x + x^3 - y\right) y \left(-1 + x^2 + x^4 - (x-1) y\right) \left(x^3 - (1-x)y\right)
\end{equation*}
of the elimination ideal $\langle p_1(x,y_1) , p_2(x,y_2), y_3 - (y_1+y_2) \rangle\cap\mathbb{K}(x)[y_3]$ and observing that none of its roots equals $\phi_1 + \phi_2$. For instance, the series $\phi$ represented by $(p,-\sqrt{2},x^2)$ is different from $\phi_1 + \phi_2$ because~$\mathrm{supp}(\phi_1)\subseteq 2 + \langle 1 \rangle$ and $\mathrm{supp}(\phi_2)\subseteq 1 + \langle -1 \rangle$ and so $1\in\mathrm{supp}(\phi_1+\phi_2)$ but $1\notin \mathrm{supp}(\phi)$ as~$\mathrm{supp}(\phi)\subseteq 2 + \langle 1 \rangle$.
The series $\phi$ encoded by $(p,\sqrt{2},x^2+2x)$ does not equal $\phi_1 + \phi_2$ because the terms of $\phi_1$ and $\phi_2$ of order at least $-\sqrt{2}\cdot 2$ and $\sqrt{2}\cdot1$ with respect to $-\sqrt{2}$ and $\sqrt{2}$ are $x^2$ and~$x$, respectively, and their sum $x^2+x$ differs from $x^2+2x$. Similar arguments apply for showing that the other series roots of $p$ do not equal $\phi_1 + \phi_2$, proving that the sum of $\phi_1$ and $\phi_2$ is not algebraic.
\end{Example}

In the following we circumvent these difficulties by simply assuming that we know a $w\in\mathbb{R}^n$ that induces a total order~$\preceq$ for which $\phi_1,\phi_2 \in\mathbb{K}_{\preceq}((\bold{x}))$ so that e.g. $\phi_1+\phi_2$ is algebraic and we can be sure to find it among the series roots of~$p(\bold{x},y)$. It is natural to do so also because this is the case in applications, for instance in the context of enumerative combinatorics~\cite[Chapter 6, Section 12]{buchacher2021algorithms}.\\
 
   
%However, a priori it is not clear how many initial terms have to be computed to decide that.  
% And of course, when $\phi_1 + \phi_2$ is algebraic, all but one of the series 

%The sum and product of two algebraic series $\phi_1$ and $\phi_2$ are algebraic if and only if there is a $w\in\mathbb{R}^n$ inducing a total order $\preceq$ on~$\mathbb{Q}^n$ with respect to which $\mathrm{supp}(\phi_1)$ and $\mathrm{supp}(\phi_2)$ have a maximal element. 

%$\phi_1,\phi_2\in\mathbb{K}_{\preceq}((x))$. 

%they are compatible in the sense that there is a field $\mathbb{K}_{\preceq}((x))$ of Puiseux series $\phi_1$ and $\phi_2$ are elements of. 

% is a $w\in\mathbb{R}^n$ inducing a total order $\preceq$ on~$\mathbb{Q}^n$ with $\phi_1,\phi_2\in\mathbb{K}_{\preceq}((x))$ only if $\phi_1 + \phi_2$ is algebraic. So to prove that $\phi_1$ and $\phi_2$ are not compatible it is sufficient to show that their sum is not algebraic which one can determine
%Let $\phi_1$ and $\phi_2$ be two series given in terms of $(m_1(X),w_1,p_1)$ and $(m_2(X),w_2,p_2)$. If $\phi_1 + \phi_2$ is algebraic, 
%a polynomial $m(X)$ that would annihilate $\phi_1+\phi_2$ and compare the initial terms of the series roots of $m(X)$ with the initial terms of $\phi_1 + \phi_2$. If the initial terms of $\phi_1 + \phi_2$ are different from the initial terms of all the series roots of $m(X)$ then $\phi_1+\phi_2$ is not algebraic and $\phi_1$ and~$\phi_2$ are not compatible. If there is only one encoding $(m(X),w,p)$ of a series root of $m(X)$ that has not been sorted out as a candidate for a representative of $\phi_1+\phi_2$ this way this does not mean that it is indeed one. However, one can verify its correctness by checking wether there are series solutions of~$m_1(X)$ and~$m_2(X)$ that are compatible with $w$ that equal $\phi_1$ and $\phi_2$, respectively. If this is the case then $\phi_1$ and~$\phi_2$ are compatible,~$\phi_1  + \phi_2$ is algebraic and $(m(X),w,p)$ is a finite encoding of it. If this is not the case, then~$\phi_1 + \phi_2$ is not algebraic, and $\phi_1$ and $\phi_2$ are not compatible.\\

Assume that $\phi_1$ and $\phi_2$ are given by $(p_1(\bold{x},y), w, q_1)$ and $(p_2(\bold{x},y),w,q_2)$. We already noted that an annihilating polynomial $p(\bold{x},y)$ for $\phi_1 + \phi_2$ can be derived from annihilating polynomials $p_1(\bold{x},y)$ and~$p_2(\bold{x},y)$ of $\phi_1$ and $\phi_2$ by computing a generator of the elimination ideal of 
\begin{equation*}
\langle p_1(\bold{x},y_1),p_2(\bold{x},y_2), y_3-(y_1+y_2)\rangle 
\end{equation*}
in $\mathbb{K}(\bold{x})[y_3]$. Whether a series root of $p(\bold{x},y)$ represented by $(p(\bold{x},y),w,p)$ equals $\phi_1 + \phi_2$ can be decided by computing the truncations $\tilde{q}_1$ and~$\tilde{q}_2$ of $\phi_1$ and $\phi_2$ up to order $\mathrm{ord}(q,w)$, where 
\begin{equation*}
\mathrm{ord}(q,w) := \min \{\langle w,\alpha\rangle \;|\; \alpha \in\mathrm{supp}(q)\}.
\end{equation*}
If $\tilde{q}_1 + \tilde{q}_2$ does not equal $q$ when ordered with respect to $w$, the series represented by $(p(\bold{x},y),w,q)$ does not equal $\phi_1+\phi_2$. However, if it does, then $(p(\bold{x},y),w,q)$ is a finite encoding of $\phi_1+\phi_2$, and by assumption we can be sure that we find a finite encoding of it this way.
%determine the terms $\tilde{p}_1$ of $\phi_1$ with respect to $w_1$ up to $\mathrm{ord}(p,w_1)$ and the terms $\tilde{p}_2$ of $\phi_2$ with respect to $w_2$ up to order $\mathrm{ord}(p,w_2)$, where 
%\begin{equation*}
%\mathrm{ord}(p,w) := \min \{\langle w,\alpha\rangle \;|\; \alpha \in\mathrm{supp}(p)\},
%\end{equation*}
%and compare $p$ with $\tilde{p}_1 + \tilde{p}_2$. If $\tilde{p}_1 + \tilde{p}_2$ does not equal $p$ when ordered with respect to $w$, then the series represented by $(m(Z),w,p)$ does not equal $\phi_1+\phi_2$.
%To check whether two series are compatible we would need to be able to compute their convex hulls and see if there is a total order with respect to which both of them have a maximal element. Since we are not able to do so, we simply assume that we know that the series are compatible and that we are given a total order they are compatible with, also because this happens to be the case in applications.\\ 
%When $\phi_1$ and $\phi_2$ are compatible in the sense that there is a total order with respect to which both of them have a leading term, then their sum $\phi_1+\phi_2$ and product are algebraic. 
%To determine a finite encoding of $\phi_1\phi_2$ we assume that we already know that $\phi_1$ and $\phi_2$ are compatible, that $w\in\mathbb{R}^n$ induces a total order $\preceq$ on $\mathbb{Q}^n$ such that $\phi_1,\phi_2\in\mathbb{K}_{\preceq}((x))$, and that $\phi_1$ and $\phi_2$ are given by $(m_1(X),w,p_1)$ and $(m(X),w,p_2)$. 

Similarly, an annihilating polynomial $p(\bold{x},y)$ for $\phi_1\phi_2$ can be determined by computing a generator of the elimination ideal of 
\begin{equation*}
\langle p_1(\bold{x},y_1),p_2(\bold{x},y_2), y_3 - y_1y_2\rangle 
\end{equation*}
in $\mathbb{K}(\bold{x})[y_3]$. To find a representation $(p(\bold{x},y),w,q)$ of $\phi_1\phi_2$ 
%it is sufficient to compare the initial terms of the series roots of $m(X)$ in $\mathbb{K}_{\preceq}((x))$ with the initial terms of $\phi_1\phi_2$. For this, 
compute the truncation $\tilde{q}_1$ of $\phi_1$ up to order~$\mathrm{ord}(q,w) - \langle w, \mathrm{lexp}_w(\phi_2)\rangle$ and the truncation $\tilde{q}_2$ of $\phi_2$ up to order $\mathrm{ord}(q,w) - \langle w, \mathrm{lexp}_w(\phi_1)\rangle$. If $q$ does not equal the sum of the first terms of $\tilde{q}_1 \tilde{q}_2$ when ordered with respect to $w$, the series represented by~$(p(\bold{x},y),w,q)$ does not equal $\phi_1\phi_2$. However, if it does, then $(p(\bold{x},y),w,q)$ is a finite encoding of $\phi_1 \phi_2$, and by assumption we can be sure that we find a finite encoding of it this way.\\

Other closure properties for algebraic series such as taking multiplicative inverses or derivatives can be performed similarly. We just refer to~\cite[Theorem 6.3]{kauers2011concrete} for an explanation of how to the corresponding annihilating polynomials can be computed.

\section*{Acknowledgement}
Part of this work was done at the Institute for Algebra of Johannes Kepler Universit\"{a}t Linz and supported by the Austrian FWF grants F5004 and P31571-N32.


\bibliographystyle{plain}
\bibliography{newtonPuiseuxAlgorithm}

% 
% 
% we first recall that its minimal polynomial can be derived from the minimal polynomials of~$\phi_1$ and $\phi_2$ by computing a generator of the elimination ideal of   
%\begin{equation*}
%\langle m_1(X), m_2(Y), Z - q(X,Y) \rangle 
%\end{equation*}
%in $\mathbb{C}(x,y)[Z]$, see~\cite{?} for details on how the theory of Gr{\"o}bner bases or resultants can be used for that. 
%%When there is no such total order, then $\phi_1 \phi_2$ is not well-defined as its coefficients involve infinite sums, and $\phi_1+\phi_2$ cannot be algebraic as the same is true for any non-linear polynomial in $\phi_1+\phi_2$. In the following we will explain how to decide whether two series are compatible and how to determine a finite representation for their sum and product.
%%Whether $\phi_1$ and $\phi_2$ are compatible can be decided by determining for each of them the convex hull of its support as explained in Subsection~\ref{sub:support}. If there is no shifted strictly-convex cone that contains their union, then neither the sum nor the product of $\phi_1$ and $\phi_2$ is algebraic, and we do not have to worry about finding an encoding for them. If this is not the case, 
%Let $m(X)$ denote the minimal polynomial of $q(\phi_1,\phi_2)$ over $\mathbb{C}[x,y]$. To determine a finite representation of $q(\phi_1,\phi_2)$ consider the finite encodings $(m(X),w,p)$ of the series roots of $m(X)$ that are compatible with $w$, i.e. encodings of the series that arise from edges $e$ of the Newton polytope of $m(X)$ for which $w\in C(e)^*$. If such a series $\phi$ does not equal $q(\phi_1,\phi_2)$ we can verify this by computing their initial terms, comparing them and observing that they are not the same. This sorts out all but one representation, which is necessarily a finite encoding of~$q(\phi_1,\phi_2)$. To verify whether $\phi$ equals $q(\phi_1,\phi_2)$ or not, we have to compute the initial terms of $q(\phi_1,\phi_2)$ up to the order $\mathrm{ord}(p,w)$ of $p$ with respect to $w$, where
%\begin{equation*}
%\mathrm{ord}(p,w) := \min \{\langle w,\alpha\rangle \;|\; \alpha \in\mathrm{supp}(p)\}.
%\end{equation*}
%To do so we compute the sum $\tilde{p}_1$ of all the terms of $\phi_1$ whose exponents $\alpha$ are such that 
%\begin{equation*}
%\langle w, (k-1)\mathrm{lexp}_w(\phi_1) + \alpha + l \mathrm{lexp}_w(\phi_2)\rangle \geq \mathrm{ord}(p,w)
%\end{equation*}
%for all $(k,l)\in\mathrm{supp}(q)$ with $k>0$, and the sum $\tilde{p}_2$ of all terms of $\phi_2$ whose exponents $\beta$ satisfy 
% \begin{equation*}
% \langle w, k\mathrm{lexp}_w(\phi_1) + (l-1)\mathrm{lexp}_w(\phi_2) + \beta \rangle \geq \mathrm{ord}(p,w).
% \end{equation*}
% for all $(k,l)\in\mathrm{supp}(q)$ with $l>0$. If the sequence of terms of $p$ does not agree with the sequence of initial terms of $q(\tilde{p}_1,\tilde{p}_2)$ when ordered with respect to $w$, then $\phi \neq q(\phi_1,\phi_2)$. If they do agree, then $\phi = q(\phi_1,\phi_2)$. 
% 
% \begin{Example}
% The series roots 
% \begin{equation*}
% \phi_1 = \sum_{n\geq0} (x+y)^n  \quad \text{and} \quad \phi_2 = \sum_{n\geq 0} \left(\frac{1}{x}+\frac{1}{y}\right)^n
% \end{equation*}
% of
% \begin{equation*}
% m_1(Z) = 1 - (1-x-y) Z \quad \text{and} \quad m_2(Z) = xy - (xy-x-y) Z 
% \end{equation*}
% encoded by 
% \begin{equation*}
% (m_1(Z),(-1+\frac{1}{\sqrt{2}},-1),1) \quad \text{and} \quad (m_1(Z),(1+\frac{1}{\sqrt{2}},1),1)
% \end{equation*}
% are not compatible because neither are
% \begin{equation*}
% \mathrm{conv}(\mathrm{supp}(\phi_1)) = \mathbb{R}_{\geq 0}^2 \quad \text{and} \quad \mathrm{conv}(\mathrm{supp}(\phi_2)) = - \mathbb{R}_{\geq 0}^2.
% \end{equation*}
%Consequently, their sum $\phi_1 + \phi_2$ and product $\phi_1\phi_2$ are not algebraic and well-defined, respectively. This can also be observed by computing a generator 
%\begin{equation*}
%m(Z) = -x - y + 2 x y - x^2 y - x y^2 + (x - x^2 + y - 3 x y + x^2 y - y^2 + x y^2) Z
%\end{equation*}
%of the elimination ideal of 
%\begin{equation*}
%\langle m_1(X), m_2(Y), Z - (X+Y) \rangle 
%\end{equation*}
%in $\mathbb{C}[x,y][Z]$ and observing that the initial terms of its series roots differ from the sum of the initial terms of $\phi_1$ and $\phi_2$.
% \end{Example}
 
%Though we do not know in advance how many initial terms we have to compute for that, we can be sure that it is only finitely many. 
%By keeping on computing initial terms and comparing them we can discard one encoding of a series root of $m(X)$ after another until there is only one left. Since we assumed that $\phi_1$ and $\phi_2$ are compatible, it is necessarily a finite representation of $q(\phi_1,\phi_2)$. 
%If we did not know already before that $\phi_1$ and $\phi_2$ are compatible we would not know whether the remaining encoding would represent $q(\phi_1,\phi_2)$ or if we had to compare further terms of $\phi$ and $q(\phi_1,\phi_2)$ to conclude that the series cannot be the same. However, if we find that none of the series solutions of $m(X)$ equals $q(\phi_1,\phi_2)$, then we have proven that $\phi_1$ and $\phi_2$ are not compatible.\\

%We now explain how the initial terms of $q(\phi_1,\phi_2)$ and $\phi$ can be computed and compared, given that we have finite encodings $(m_1(X),w_1,p_1)$, $(m_2(X),w_2,p_2)$ and $(m(X),w,p)$ of $\phi_1$, $\phi_2$ and $\phi$, respectively. To compute the terms of $q(\phi_1,\phi_2)$ up to e.g. the order $\mathrm{ord}(p,w)$ of $p$ with respect to $w$, we determine the orders $\mathrm{ord}(p,w_1)$ and $\mathrm{ord}(p,w_2)$ with respect to $w_1$ and $w_2$, where  
%\begin{equation*}
%\mathrm{ord}(p,w) := \min \{\langle w,\alpha\rangle \;|\; \alpha \in\mathrm{supp}(p)\}, 
%\end{equation*}
%and compute the sum $\tilde{p}_1$ of all terms of $\phi_1$ whose exponents $\alpha$ satisfy
%\begin{equation*}
%\langle w_1, \alpha + \mathrm{lexp}(p_1) + l \mathrm{lexp}(p_2) \rangle 
%\end{equation*}
%\\



%observe this by comparing the initial terms of By comparing the initial terms of $\phi$ with the initial terms of $q(\phi_1,\phi_2)$ we can discard those 

 %for instance, choose an element $w\in\mathbb{R}^2$ that induces a total order compatible with both of $\phi_1$ and $\phi_2$ and then determine the encodings $(m(X),w,p)$ of the series solutions of $m(X)=0$ that result from those edges $e$ of its Newton polytope for which $w\in C^*(e)$. Since (precisely) one of these encodings represents $\phi_1+\phi_2$ we just have to identify those for which this is not the case. 
%first determining for each series solution of $m(X)=0$ an encoding $(m(X),\tilde{w},\tilde{p}_i)$. Since $\phi_1$ and $\phi_2$ are compatible, there is necessarily one for the sum of $\phi_1$ and $\phi_2$ among them, and we can identify it by determining the encodings of those series that differ from $\phi_1+\phi_2$. 
%If it happens that all of them represent series different from $\phi_1+\phi_2$, then the sum of $\phi_1$ and $\phi_2$ is not algebraic. If there is only one left for which we cannot show that it encodes $\phi_1+\phi_2$, then the sum of $\phi_1$ and $\phi_2$ is algebraic, and it is an encoding for it. 
%To prove that a series solution of $m(X)$ given by $(m(X),w,p)$ does not equal $\phi_1+\phi_2$, let $o_1$ and $o_2$ be the orders of $p$ with respect to $w_1$ and $w_2$, respectively, and assume that the order of the trailing term of $p_1$ is at most~$o_1$ with respect to $w_1$ and that the one of  $p_2$ is at most $o_2$ with respect to $w_2$. If the initial terms of $p_1+p_2$ do not agree with those of $p$ when ordered with respect to $w$, then $(m(X),w,p)$ cannot be the encoding of~$\phi_1+\phi_2$. 
%If however, the initial terms of $p_1+p_2$ do agree with those of $\tilde{p}_i$ then $(m(X),\tilde{w}_i,\tilde{p}_i)$ might be a representation of $\phi_1+\phi_2$, but it might as well be not. To be sure that it is not, in case it is not, let $\tilde{\phi}$ be the series encoded by $(m(X),\tilde{w}_i,\tilde{p}_i)$ and let~$C$ be the cone output by Algorithm~\ref{alg:NPA} such that~$\mathrm{supp}(\tilde{\phi})\subseteq\{\alpha_1,\dots,\alpha_{N-1}\}\cup (\alpha_N+C)$ where $\alpha_1,\dots,\alpha_N$ are the exponents of $\tilde{p}_i$. Assume that $C$ is minimal, and furthermore assume that the order of $\tilde{p}_i$ equals the minimum of the orders of $p_1$ and $p_2$ with respect to $\tilde{w}_i$ and that the order of $p_1$ is smaller than the order of $p_2$ with respect to $w_1$. Then 
%Assuming that $C$ is minimal 
%
%\newpage
%
%WORK IN PROGRESS. OLD STUFF AHEAD
%
%
%
%We are now able to answer the last of our three questions raised in this section by addressing the following problem: Given a series $\phi$ in $\mathbb{C}_{\preceq}((x,y))$, represented by its minimal polynomial $m(X)$ over $\mathbb{C}[x,y]$, a total order $\preceq$ and its first terms with respect to it, and given a polynomial $p(X) = \sum_i p_i X^i$ over $\mathbb{C}(x,y)$, determine the leading exponent $\alpha$ of $p(\phi)$ with respect to $\preceq$ and a cone $C$ such that $\mathrm{supp}(p(\phi))\subseteq \alpha + C$ when $p(\phi)$ is viewed as an element of $\mathbb{C}_{\preceq}((x,y))$. We first note that, since $\phi$ is algebraic, and $p(X)$ is a polynomial, also their composition $p(\phi)$ is algebraic, and the minimal polynomial of $p(\phi)$ can be determined from $p(X)$ and the minimal polynomial $m(X)$ of $\phi$ by computing a generator of the elimination ideal 
%\begin{equation*}
%\langle m(X), Y - p(X) \rangle \cap \mathbb{C}(x,y)[Y]
%\end{equation*}
%using Gr{\"o}bner bases, see~\cite{kreuzer2000}, for instance. Let $m_{p(\phi)}(X)$ denote the minimal polynomial of $p(\phi)$ over $\mathbb{C}[x,y]$. Using Algorithm~\ref{alg:NPA} we can determine the first terms of the series solutions of $m_{p(\phi)}(X)=0$ that are compatible with $\preceq$, i.e. that have a leading term with respect to $\preceq$, and check which of them equals $p(\phi)$ as an element of $\mathbb{C}_{\preceq}((x,y))$ by comparing their initial terms with those of $p(\phi)$. To explain how to determine the first terms of $p(\phi)$ assume that $w\in\mathbb{R}^2$ is a vector which induces $\preceq$. Given a Laurent polynomial $q(x,y)\in\mathbb{C}_{\preceq}((x,y))$, we denote the weight of its leading term by $w_+(q)$ and the weight of its trailing term by $w_-(q)$, i.e.~$w_+(q)$ is the Euclidean product of $w$ with the leading exponent of $q$, while~$w_-(q)$ is the product of $w$ and the trailing exponent of $q$, if there is a trailing exponent at all. To compute the terms of $p(\phi)$ whose weight is at least $c_0\in\mathbb{R}$, determine the sum~$\bar{\phi}$ of first terms of $\phi$ until $w_+(p_i) + (i-1) w_+(\bar{\phi}) + w_-(\bar{\phi}) < c_0$ and the sum $\bar{p}_i$ of first terms of $p_i$ until $w_-(\bar{p}_i) + i w_+(\phi) < c_0$, for $i>0$, and the sum $\bar{p}_0$ of first terms of~$p_0$ whose weight is not smaller than $c_0$. The terms of $p(\phi)$ whose weight is at least $c_0$ are the terms of $\sum_i \bar{p}_i \bar{\phi}^i$ for which this is true. We illustrate our reasoning with an example.
%
%\begin{Example}\label{ex:comp}
%Let $\preceq$ be the total order on $\mathbb{Q}^2$ induced by $w = (-\sqrt{2},-1)$, and let $\phi$ be the series solution of $m(X) = 1 + x - X^2 = 0$ in $\mathbb{C}_{\preceq}((x,y))$ whose first term is $1$. Furthermore, let
%\begin{equation*}
%p(X) = \frac{x+y}{1-y}+(x+y)X.
%\end{equation*}
%%To compute the leading exponent $\alpha$ of $p(\phi)$ and a cone $C$ such that $\mathrm{supp}(\phi)\subseteq \alpha + C$ when viewed as an element of $\mathbb{C}_{\preceq}((x,y))$, we first observe that 
%The minimal polynomial of $p(\phi)$ is
%\begin{equation*}
%m_{p(\phi)}(X) = (x + y)^2 (x (-1 + y)^2 + (-2 + y) y) + 2 (x + y) (1-y) X - (1-y)^2 X^2.
%\end{equation*}
%Its Newton polytope has $8$ admissible edges, two of which give rise to series solutions in $\mathbb{C}_{\preceq}((x,y))$. One of these edges is $\{(0,0,2),(0,1,1)\}$. The first term of the corresponding series is $2y$,  and its support is contained in $(0,1) + \langle (0,1),(1,-1) \rangle$. The other edge is $\{(0,1,1),(0,3,0)\}$, the first term of the resulting series is $y^2$, and its support is contained in $(0,2) + \langle (1,0),(1,-1) \rangle$. Since the first term of~$p(\phi)$ in~$\mathbb{C}_{\preceq}((x,y))$ is $2y$, we now know that 
%\begin{equation*}
%\mathrm{supp}(p(\phi))\subseteq (0,1) + \langle (0,1),(1,-1) \rangle.
%\end{equation*}
%A similar reasoning as in Example~\ref{ex:nonUniqueness} shows that the series constructed from the edge $\{(0,0,2),(1,0,1)\}$ of the Newton polytope of $m_{p(\phi)}(X)$ equals $p(\phi)\in\mathbb{C}_{\preceq}((x,y))$, so that also
%\begin{equation*}
%\mathrm{supp}(p(\phi))\subseteq (1,0) + \langle (1,0),(-1,1) \rangle.
%\end{equation*}
%\end{Example}
%
%
%\part{The Orbit-Sum Method}
%
%We begin with introducing discrete differential equations, explain how they arise in enumerative combinatorics and  illustrate the orbit-sum method by solving a partial discrete differential equation of order~$1$.
%
%\begin{Definition}
%Let $F\in\mathbb{K}[x,y][[t]]$. The discrete derivative $\Delta_x F$ of $F$ with respect to $x$ is defined by 
% \begin{equation*}
%  \Delta_x F(x,y;t) := \frac{F(x,y;t)-F(0,y;t)}{x}.
% \end{equation*}
%Its discrete derivative $\Delta_x^k F$ of order $k\in\mathbb{N}$ with respect to $x$ is defined recursively by $\Delta_x^0 F := F$ and $\Delta_x^k F := \Delta_x\left(\Delta_x^{k-1} F\right)$ for $k>0$. The analogous statement holds for higher discrete derivatives with respect to $y$.
%\end{Definition}
%
%\begin{Definition} 
%Let $P\in\mathbb{K}[x,y]$ and $Q\in\mathbb{K}[v_0,\dots,v_{k+l},x,y,t]$ and $k,l\in\mathbb{N}$. An equation of the form 
%\begin{equation}\label{eq:DDE}
% F = P(x,y) + t Q(F,\Delta_x F,\dots,\Delta_x^{k} F, \Delta_y F,\dots,\Delta_y^{l} F,x,y,t)
%\end{equation}
%for (unknown) $F\in\mathbb{K}[x,y][[t]]$ is called a discrete differential equation (DDE). It is called a partial discrete differential equation (PDDE) when it involves a discrete derivative of $F$ with respect to $x$ and $y$, otherwise it is said to be an ordinary discrete differential equations (ODDE). When the total degree of $P$ with respect to $v_0,\dots,v_{k+l}$ is at most $1$, then the DDE is said to be linear, otherwise non-linear.     
%\end{Definition}
%
%A DDE has a unique solution in $\mathbb{K}[x,y][[t]]$. Its initial terms can be determined by using the recurrence that results from extracting the coefficient of $t^n$ in equation~\eqref{eq:DDE}. The problem related to DDE's is the following.  
%\begin{Problem}\label{prob:class}
% Given a DDE, decide whether its solution is algebraic, D-finite or D-algebraic, and in case it is, determine a polynomial or (linear) differential equation satisfied by it.
%\end{Problem}
%
%DDE's arise in the enumeration of maps~\cite{?} and lattice walks restricted to convex cones~\cite{?}. We recall the notion of a lattice walk, present a simple example of a partial DDE, and explain how it can be solved via the orbit-sum method.
%
%\begin{Definition}
%A lattice walk is a sequence $P_0,\dots,P_n$ of points of $\mathbb{Z}^2$. We call $P_0$ and $P_n$ its starting and end point, respectively, the consecutive differences $P_{i+1}-P_i$ are its steps, and $n$ is its length. 
%\end{Definition}
%Given $S\subseteq \mathbb{Z}^2$ and $(i,j)\in\mathbb{Z}^2$ and $n\in\mathbb{N}$, denote by $f(i,j;n)$ the number of walks in $\mathbb{N}^2$ that start at the origin, end at $(i,j)$ and consist of precisely $n$ steps all of which are taken from $S$. Many questions about $f(i,j;n)$
%%\begin{Problem}
%%Given $S\subseteq \mathbb{Z}^2$ and $(i,j)\in\mathbb{Z}^2$ and $n\in\mathbb{N}$, what is the number $f(i,j;n)$ of lattice walks in $\mathbb{N}^2$ that start at the origin, end at $(i,j)$ and consist of precisely $n$ steps all of which are taken from $S$? 
%%\end{Problem}
%%The question 
%can be approached by considering the associated generating function 
%\begin{equation*}
% F(x,y;t) := \sum_{n\geq 0} \left( \sum_{i,j\geq 0} f(i,j;n) x^iy^j \right) t^n 
% %\in\mathbb{Q}[x,y][[t]],
%\end{equation*}
%studying the DDE that reflects the recursive construction of lattice walks, and solving Problem~\ref{prob:class} for it.
%
%\begin{Example}
%Let $S = \{(1,0), (0,1), (-1,0), (0,-1)\}$ and define the corresponding step polynomial $S(x,y) := x + y + x^{-1} + y^{-1}$. As any lattice walk is either of length~$0$ or a walk followed by a step from $S$, unless it ends on the $x$-axis, where it can not be extended by a down-step, and unless it ends on the $y$-axis, where it can not be continued by a left-step, the functional equation for $F$ is
%\begin{equation*}\label{eq:DDE2}
% F(x,y;t) = 1 + t (x+y) F(x,y;t) + t \frac{F(x,y;t)-F(x,0;t)}{y} + t \frac{F(x,y;t)-F(0,y;t)}{x}.
%\end{equation*}
%\end{Example}
%The previous functional equation is an example of a linear partial DDE of order~$1$. Its solution in $\mathbb{Q}[x,y][[t]]$ is D-finite as we now show~\cite{?}. 
% \begin{Example}\label{ex:OSmethod}
% Reordering the terms of equation~\eqref{eq:DDE2} and multiplying it by $x y$ gives 
% \begin{equation}\label{eq:DDE3}
%  x y (1-t S) F(x,y) = xy - tx F(x;0) - ty F(y;0),
% \end{equation}
% where we write $S\equiv S(x,y)$ and $F(x,y)\equiv F(x,y;t)$ for notational convenience. These equations have the property that they are seemingly underdetermined: there is only one equation, but there are several unknowns. However, we can derive further equations by performing substitutions. By repeatedly replacing $x$ and $y$ in equation~\eqref{eq:DDE3} by $\bar{x}:=x^{-1}$ and $\bar{y}:=y^{-1}$, respectively, we find three additional equations,
% \begin{align*}
%  \bar{x} y (1-t S) F(\bar{x},y) &= \bar{x}y - t \bar{x} F(\bar{x};0) - ty F(0,y),\\
%  \bar{x} \bar{y} (1-t S) F(\bar{x},\bar{y}) &= \bar{x}\bar{y} - t \bar{x} F(\bar{x};0) - t\bar{y} F(0,\bar{y}),\\
%  x \bar{y}(1-t S) F(x,\bar{y}) &= x\bar{y} - t x F(x;0) - t\bar{y} F(0,\bar{y}). 
% \end{align*}
% Note that the step polynomial $S$ is not altered by these substitutions, and that for each evaluation of $F$ on the right-hand side of these equations, there is another equation that involves the same evaluation. These equations can now be linearly combined such as to eliminate all evaluations of $F$,
% \begin{equation*}
%  xy F(x,y) - \bar{x}y F(\bar{x},y) + \bar{x}\bar{y} F(\bar{x},\bar{y}) - x\bar{y} F(x,\bar{y}) =  \frac{xy -\bar{x}y + \bar{x}\bar{y} - x\bar{y}}{1-t S}. 
% \end{equation*}
% Since $xyF(x,y)$ only involves positive powers in $x$ and $y$, and because all the other terms on the left hand-side of this equation involve either a negative power in $x$ or a negative power in $y$, we find that
% \begin{equation*}
%  xy F(x,y) = [x^>y^>] \frac{xy -\bar{x}y + \bar{x}\bar{y} - x\bar{y}}{1-t S},
% \end{equation*}
% where $[x^>y^>]$ is the operator that acts on series by discarding all terms that involve a non-positive power in $x$ or $y$.
%\end{Example}
%
%The orbit-sum method consists of three steps: 
%\begin{enumerate}
% \item deriving additional equations by performing certain algebraic substitutions,
% \item linearly combining them such as to eliminate the evaluations of the unknown,
% \item finding an expression of the unknown by applying $[x^>y^>]$.
%\end{enumerate}
% 
%Though each of these steps appears to be particularly simple in the above example, they become more complicated when the orbit-sum method is applied to PDDE's whose order is greater than~$1$. The following sections present the complications that arise and explain how they can be approached. 
%
%\section{Orbits and Orbit-Equations}
%
%%Very often there is no (suitable) group that can be associated with a model which could be used to find a functional equation for its generating function that does not involve any of its sections. 
%The problem of determining suitable substitutions was discussed in~\cite{large} and addressed by introducing the so-called orbit.
%%the group associated with a model does not allow to find a functional equation not involving any of the sections of $F_0$ or $F_1$. For some of these models the reason is that the group does not capture enough information about the model. This problem was discussed in~\cite{mireille} and solved by replacing the group by the so-called orbit of the model. 
%For convenience we recall it here.
%
%The substitutions we used to solve equation~\eqref{eq:DDE2} had the following property: with any substitution$(x',y')$, there were other substitutions $(x'',y'')$ and $(x''',y''')$ such that $x' = x''$ and $S(x',y') = S(x'',y'')$, and $y' = y'''$ and $S(x',y') = S(x''',y''')$. They allowed to modify equation~\eqref{eq:DDE2} without altering~$S(x,y)$, and without altering one of the unknown evaluations of $F$. The orbit of a Laurent polynomial is a set of algebraic functions that formalizes these properties. 
%\begin{Definition}
%Given a polynomial $p$ in $\mathbb{Q}[x,y,x^{-1},y^{-1}]$, let $\sim$ be the relation on~$\overline{\mathbb{Q}(x,y)}^2$ defined by  
%\begin{equation*}
%(u_1,u_2)\sim (v_1,v_2)\quad  :\Longleftrightarrow \quad u_1 = v_1 \text{ or } u_2 = v_2, \text{ and } p(u_1,u_2) = p(v_1,v_2),
%\end{equation*}
%and let $\approx$ be its transitive closure. The orbit of $p$ is the set of elements of $\overline{\mathbb{Q}(x,y)}^2$ which are equivalent to $(x,y)$.
%\end{Definition}
%
%In general one can not hope for explicit expressions of the elements of the orbit in terms of radicals, and even if there are such expressions, it might not be efficient to compute with them. Instead, we work with their minimal polynomials. The following algorithm determines them using resultants, see also~\cite{large}
%%describes how the minimal polynomials of the elements of an orbit can be computed via resultants.
%\begin{Algorithm}\label{alg:4}
%  Input: A polynomial $p$ in $\mathbb{Q}[x,y]$.\\
%  Output: The set of (pairs of) minimal polynomials of elements of the orbit of $p$. 
%  \step 10 Set $p = p(x,y) - p(X,Y)$, and $\mathrm{done} = \emptyset$, and $\mathrm{todo} = \{(-X+x,-Y+y)\}$.
%  \step 20 While $\mathrm{todo}\neq \emptyset$, do:
%  \step 31 Remove an element $(P_{old},Q_{old})$ from $\mathrm{todo}$ and add it to $\mathrm{done}$.
%  \step 41 Compute the set $P_{new}$ of irreducible factors of the resultant $\mathrm{res}_Y(Q_{old},p)$ of $Q_{old}$ and $p$ with respect to $Y$, which are not free of and not equal to $Y$.
%  \step 51 Compute the set $Q_{new}$ of irreducible factors of the resultant $\mathrm{res}_X(P_{old},p)$ of $P_{old}$ and $p$ with respect to $X$ which are not free of and not equal to $X$.
%  \step 61 Enlarge $\mathrm{todo}$ by the pairs consisting of $P_{old}$ and the elements of $Q_{new}$, and of elements of $P_{new}$ and of $Q_{old}$ unless they are elements of $\mathrm{done}$.
%  \step 70 Return $\mathrm{done}$.
%\end{Algorithm}
%% 
%% \begin{Proposition}
%% For $546$ models there is no group that can associated with them. At least $95$ of them have a finite orbit. For the remaining $451$ models we do not know whether their orbit is finite or not.
%% \end{Proposition}
%% 
%
%
%The components of the elements of the orbit lie in the splitting field $L$ of the set of their minimal polynomials over $K = \mathbb{Q}(x,y)$. If the algorithm terminates then the orbit is finite and $L$ is generated by finitely many elements that are algebraic over $K$, i.e. $L/K$ is a finite extension of $K$. Since $K$ has characteristic zero $L/K$ is separable, and so the primitive element theorem implies that there is an element~$\alpha$ of $L$ such that $L = K(\alpha)$. Let $m(X)$ denote the minimal polynomial of $\alpha$ over $K$, and let $d$ be its degree. The evaluation at $\alpha$ induces an isomorphism between~$K[X]/\langle m(X)\rangle$ and $K(\alpha)$ and allows to identify elements of $K(\alpha)$ with polynomials of $K[X]$ whose degree is smaller than $d$, their so-called canonical representatives. Note that computations in $K[X]/\langle m(X)\rangle$ just amount to adding and multiplying polynomials, performing division with remainder and computing modular inverses using the extended Euclidean algorithm. 
%%We explain how the field operations are performed on the level of canonical representatives. Let $p_1,p_2$ be polynomials in $K[x]$ with $\deg p_1, \deg p_2 < d$. The canonical representative of $p_1(\alpha)+p_2(\alpha)$ is just $p_1+p_2$, and the canonical representative of $p_1(\alpha)p_2(\alpha)$ is $\mathrm{rem}(p_1p_2,m)$, the remainder of $p_1p_2$ when divided by $m$. The canonical representative of $1/p_1(\alpha)$ is determined by using the extended Euklidean algorithm to find polynomials $s$ and $t$ in $K[x]$ such that $p_1 s + m t = 1$, and computing the remainder $\mathrm{rem}(s,m)$. 
%
%A primitive element $\alpha$ of the splitting field $L/K$ of a set of polynomials over $K$ and expressions of their roots in terms of $\alpha$ can be found using Gr{\"o}bner bases and the so-called Shape Lemma, see~\cite[Theorem~3.7.25]{kreuzer2000}.
%
%\begin{Definition}
%Let $I\subseteq K[x_1,\dots,x_n]$ be a zero-dimensional ideal. It is said to be in normal $x_i$-position, $i\in\{1,\dots,n\}$, if any two common zeros $(a_1,\dots,a_n)$ and~$(b_1,\dots,b_n)$ of $I$ in $\overline{K}^n$ satisfy $a_i\neq b_i$.
%\end{Definition}
%
%\begin{Theorem}\label{theorem:prime}(Shape Lemma)
%Let $K$ be a perfect field, $I\subseteq K[x_1,\dots,x_n]$ a zero-dimensional radical ideal in normal $x_n$-position, $g_n\in K[x_n]$ the monic generator of the elimination ideal $I\cap K[x_n]$ and $d$ its degree. Then a Gr{\"o}bner basis of $I$ is of the form $\{x_1-g_1,\dots ,x_{n-1}-g_{n-1},g_n\}$ for $g_1,\dots,g_{n}\in K[x_n]$. In particular, the common roots of $I$ are $(g_1(a),\dots,g_{n-1}(a),a)$ where $a\in\overline{K}$ is a root of~$g_n$.
%\end{Theorem}
%
%Assume that $L$ is the splitting field of a set $\{p_1(X),\dots, p_n(X)\}$ of monic, irreducible and pairwise distinct polynomials over $\mathbb{Q}(x,y)$. Let $I$ be the ideal generated by $p_{i}(X_{ij})$ for $i=1,\dots,n$ and $j=1,\dots,\deg_X(p_i)$ and $q_1 = 1- t \prod_{ij\neq kl} (X_{ij}-X_{kl})$ and $q_2 =  Z - \sum_{ij} a_{ij}X_{ij}$, where the $X_{ij}$'s and $Z$ and $t$ are variables and the $a_{ij}\in\mathbb{Q}$ are such that $I$ is in normal $Z$-position, see~\cite[Definition~3.7.21]{kreuzer2000}. It is no restriction to assume that all the assumptions of the theorem are satisfied: $\mathbb{Q}(x,y)$ has characteristic zero and hence it is a perfect field, the set of common zeros of the elements of $I$ is finite therefore $I$ is zero-dimensional and if $I$ is not radical we simply simply replace it by its radical $\sqrt{I}$ without altering the set of common roots, see~\cite[Corollary~3.7.16]{kreuzer2000}. 
%%TO-DO: check whether our ideal is always radical.
%
%%Consequently, the Gr{\"o}bner basis of $I$ with respect to lexicograhic order provides polynomials $g_{ij}$ and $g$ in $\mathbb{Q}(x,y)[z]$ such that any root $\alpha$ of $g$ is a primitive element for $L/K$ and the roots $\alpha_{ij}$ of $p_i$ are polynomials in $\alpha$ given by $\alpha_{ij} = g_{ij}(\alpha)$.\\
%
%\begin{Algorithm}\label{alg:5}
%  Input: A set $\{p_1(X),\dots,p_n(X)\}$ of monic, irreducible and pairwise distinct polynomials over $\mathbb{Q}(x,y)$.\\
%  Output: The minimal polynomial of a generator of the splitting field of these polynomials over $\mathbb{Q}(x,y)$ and a representation of their roots as polynomials in this generator over $\mathbb{Q}(x,y)$.
%  \step 10 Let $d_i=\deg_X p_i$, and let $Z$, $t$ and $X_{ij}$ for $i\in\{1,\dots,n\}$ and $j\in\{1,\dots,d_i\}$ be variables.
%  \step 20 Define $q_1 = 1 - t\prod_{i=1}^n \prod_{1\leq j_1 < j_2\leq d_i} (X_{ij_1}-X_{ij_2})$.
%  \step 30 Define a polynomial $q_2 = Z - \sum_{i=1}^n \sum_{j=1}^{d_i} a_{ij} X_{ij}$ with random integer coefficients $a_{ij}$.
%  \step 40 Compute a Gr{\"o}bner basis of the ideal 
%  \begin{equation*}
%  I = \langle p_i(x_{ij}) \mid i\in\{1,\dots,n\}, j\in\{1,\dots, d_j\} \rangle + \langle q_1,q_2 \rangle
%  \end{equation*}
%  in the ring of polynomials in $Z, t$ and the $X_{ij}$'s over $\mathbb{Q}(x,y)$ with respect to a lexicographic order where $t$ is the largest and $Z$ the smallest variable, if it is radical, otherwise do it for its radical.
%  \step 50 Repeat steps $3$ and $4$ until the Gr{\"o}bner basis contains a non-zero polynomial $g$ in $\mathbb{Q}(x,y)[z]$ and polynomials of the form $X_{ij} - g_{ij}$ with $g_{ij}$ in $\mathbb{Q}(x,y)[z]$ for each $i$ and $j$.
%  \step 60 Return $g$ and the $g_{ij}$'s.
%\end{Algorithm}
%
%
%Given the (finite) orbit of a polynomial in terms of the minimal polynomials of its elements, Algorithm~\ref{alg:5} allows to find the minimal polynomial of a generator~$\alpha$ of their splitting field over $\mathbb{Q}(x,y)$ as well as canonical representatives of the elements of the orbit in terms of polynomials in $\alpha$. We note that, if $(p_1,p_2)$ is a pair of minimal polynomials of an element of the orbit of a polynomial $p(x,y)$, and~$\alpha_1$ and $\alpha_2$ denote roots of $p_1$ and $p_2$, respectively, then $p(\alpha_1,\alpha_2)$ need not equal $p(x,y)$, i.e. $(\alpha_1,\alpha_2)$ need not be an element of the orbit of $p(x,y)$. So this has to be checked additionally. We also note that the size of the polynomials in the output of Algorithm~\ref{alg:5} is sensitive to the choice of $q_2$ defined in step $3$ of the algorithm. The problem of finding primitive elements that are nice in this respect is discussed in~\cite{van2008}.
%
%%\section{Orbit-Equations}
%%
%%Given a model $(S_0,S_1)$ whose Minkowski sum can be associated with a finite orbit Algorithm~\ref{alg:6} determines the vector space of functional equations for $F_0$ and $F_1$ which do not involve any of their sections.
%%
%%\begin{Algorithm}\label{alg:6}
%%  Input: A pair $(S_0,S_1)$ of step sets whose orbit is finite.\\
%%  Output: A basis of a vector space of equations for $F_0$, which neither involve sections of $F_0$ nor of $F_1$.
%%  \step 10 Set up the system of functional equations for $F_0$ and $F_1$.
%%  \step 20 Eliminate $F_1$ from this system.
%%  \step 30 Compute the orbit of the step polynomial $S(x,y)$ of the Minkowski sum of $S_0$ and $S_1$.
%%  \step 40 Replace $(x,y)$ in the equation computed in step $2$ by the elements of the orbit.
%%  \step 50 Form a linear combination of the equations computed in step $4$ with undetermined coefficients and set up a linear system by setting coefficients of sections of $F_0$ and $F_1$ equal to zero.
%%  \step 60 Compute a basis of the vector space of solutions of the resulting linear system.
%%  \step 70 Return a basis of the vector space of section-free equations for $F_0$.
%%\end{Algorithm}
%%
%%Example~\ref{ex:noCancelling} shows that it is not always possible to eliminate all the sections.
%%
%%\begin{Example}\label{ex:noCancelling}
%%Let $(S_0,S_1)$ be given by $S_0 = \{(-1,0),(0,1)\}$ and $S_1=\{(0,0),(1,-1)\}$. Their Minkowski sum is $S=\{(-1,0),(0,-1),(0,1),(1,0)\}$ and the group $G$ that is associated with it is
%%\begin{equation*}
%%G = \{(x,y),(\bar{x},y),(\bar{x},\bar{y}),(x,\bar{y})\}.
%%\end{equation*}
%%Eliminating $F_1$ from the system of equations for $F_0$ and $F_1$ results in
%%\begin{equation*}
%%(1-t^2S)xyF_0 = xy - tx^2F_1(x,0) - t^2(x+y)F_0(0,y).
%%\end{equation*}
%%Replacing $(x,y)$ by the elements of $G$ gives four equations, 
%%%\begin{align*}
%%%(1-t^2S)xyF_0(x,y) &= xy - tx^2F_1(x,0) - t^2(x+y)F_0(0,y)\\
%%%(1-t^2S)\bar{x}yF_0(\bar{x},y) &= \bar{x}y - t\bar{x}^2F_1(\bar{x},0) - t^2(\bar{x}+y)F_0(0,y)\\
%%%(1-t^2S)\bar{x}\bar{y}F_0(\bar{x},\bar{y}) &= \bar{x}\bar{y} - t\bar{x}^2F_1(\bar{x},0) - t^2(\bar{x}+\bar{y})F_0(0,\bar{y})\\
%%%(1-t^2S)x\bar{y}F_0(x,\bar{y}) &= x\bar{y} - tx^2F_1(x,0) - t^2(x+\bar{y})F_0(0,\bar{y}),
%%%\end{align*}
%%which cannot be non-trivially linearly combined to cancel the sections. The same is true when working with the equation resulting from eliminating $F_0$.
%%\end{Example}
%%
%%\begin{Example}\label{ex:noCancelling1}
%%Let $(S_0,S_1)$ be the model with step sets $S_0=\{(-1,-1),(0,0),(1,1)\}$ and $S_1=\{(-1,1),(0,0),(1,-1)\}$. As before, their Minkowski sum can be associated with a finite group, but there is no linear combination of the equations of the respective orbits which does not involve any section of $F_0$ and $F_1$.
%%\end{Example}
%%
%%% For $334$ of the $418$ models which have a (finite) group we can determine the vector space of section-free orbit equations, and so we can for $53$ of the at least $95$ models which do not have a group but a finite orbit. For $87$ models we could not determine the vector space of section-free orbit equations, either because it was too hard to determine a Gr{\"o}bner basis to find a primitive element of the splitting field of the minimal polynomials of the elements of the group / orbit or because their representations were to large to solve the subsequent linear system in a reasonable amount of time.\\
%%% 
%%%Sometimes it is not clear whether a model is rational, algebraic or D-finite although one can solve the associated functional equation and find an expression for its generating functions.
%%In Example~\ref{ex:noCancellingSolved} we solve the functional equation of the model from Example~\ref{ex:noCancelling} and determine an expression for its generating function. However, from the given expression we are not able to draw any conclusions about its nature, i.e. whether it is rational, algebraic, D-finite or D-algebraic.
%%%About how to find an expression of a generating function we cannot say much about.
%%
%%\begin{Example}\label{ex:noCancellingSolved}
%%We continue with Example~\ref{ex:noCancelling}. Let $Y_0$ be the root of $xy(1-t^2S)$ which lies in $\mathbb{Q}[x,\bar{x}][[t]]$ when viewed as a polynomial in $y$. Replace $(x,y)$ by $(x,Y_0)$ and $(\bar{x},Y_0)$ in
%%\begin{equation*}
%%(1-t^2S)xyF_0 = xy - tx^2F_1(x,0) - t^2(x+y)F_0(0,y)
%%\end{equation*}
%%to find
%%\begin{align*}
%%0 &= xY_0 - tx^2F_1(x,0) - t^2(x+Y_0)F_0(0,Y_0)\\
%%0 &= \bar{x}Y_0 - t\bar{x}^2F_1(\bar{x},0) - t^2(\bar{x}+Y_0)F_0(0,Y_0).
%%\end{align*}
%%Eliminating $F_0(0,Y_0)$ from these equations results in
%%\begin{equation*}
%%x^2 F_1(x,0) = \frac{(x^2-1)Y_0^2}{t(1+xY_0)} + \frac{x+Y_0}{x(1+xY_0)} F_1(\bar{x},0).
%%\end{equation*}
%%The coefficient of $F_1(\bar{x},0)$ can be written as the product of two D-algebraic series~$f_-$ and $f_+^{-1}$ such that $f_-\in\mathbb{Q}[\bar{x}][[t]]$ and $f_+\in\mathbb{Q}[x][[t]]$. Consequently,
%%\begin{equation*}
%%F_1(x,0) = \bar{x}^2 f_+^{-1} [x^>]f_+ \frac{(x^2-1)Y_0^2}{t(1+xY_0)}.
%%\end{equation*}
%%%Since the class of D-algebraic functions is closed under addition and multiplication as well as under taking positive part, see \cite{Hadamard prod of power series}, $F_1$ and $F_0$ are D-algebraic.
%%\end{Example}
%%
%%Following an idea from~\cite{Gessel} we explain how to find such a factorization. Given a series $f\in\mathbb{Q}[x,\bar{x}][[t]]$ we can write 
%%\begin{align*}
%%f &= e^{\log f}\\ 
%%  &= e^{[x^\leq] \log f + [x^>]\log f}\\ 
%%  &= e^{[x^\leq] \log f}e^{[x^>]\log f}\\ 
%%  &= e^{[x^\leq] \log f}(e^{-[x^>]\log f})^{-1},
%%\end{align*}
%%unless the lowest order term of $f$ with respect to $t$ depends on $x$. So $f=f_- f_+^{-1}$ with $f_- = e^{[x^\leq] \log f}$ and $f_+ = e^{-[x^>]\log f}$, and by construction $f_-\in\mathbb{Q}[\bar{x}][[t]]$ 
%%and $f_+\in \mathbb{Q}[x][[t]]$. When $f$ is algebraic, then $f_-$ and $f_+$ are (at least) differentiably algebraic, see~\cite{Kilian} for closure properties of differentiably algebraic and D-finite functions. It is not clear whether $F_1(x,0)$ is D-algebraic. In~\cite{hypergeometric} it was explained that the positive part of a series can be encoded as the Hadamard product with a rational series. In~\cite{sharif} it was proven that the Hadamard product of the series expansion of a univariate rational function and a univariate D-algebraic power series is D-algebraic while in general this is not true for multivariate series. But it is not clear whether the latter is also true for the Hadamard product of the series expansion of a rational function in a single variable $x$ and a series in $\mathbb{Q}[x,\bar{x}][[t]]$.
%
%\section{Positive-Part-Extraction}
%
%The previous section recalled how the minimal polynomials of the algebraic substitutions required by the orbit-sum method are determined, and explained how Gr\"{o}bner bases and the Shape Lemma allow to reduce computations in their splitting field to simple polynomial arithmetic. As a consequence step~$(1)$ and step~$(2)$ of the orbit-sum method can be performed automatically, the result of the computations being a basis of the vector space of section-free orbit-equations whose elements are of the form 
%\begin{equation}\label{eq:orbitEquation}
%xyF(x,y;t) + \sum_{p_1,p_2,p_3} p_3(\alpha) F(p_1(\alpha),p_2(\alpha);t) = p(\alpha),
%\end{equation}
%where $F\in\mathbb{Q}[x,y][[t]]$ is unknown, $\alpha$ is an element of $\overline{\mathbb{Q}(x,y)}$, given by its minimal polynomial over$\mathbb{Q}[x,y]$, and~$p_1(\alpha), p_2(\alpha)$ and $p_3(\alpha)$ and $p(\alpha)$ are polynomials in $\alpha$ over $\mathbb{Q}(x,y)$.
%
%The purpose of this section is to give a meaning to 
%\begin{equation*}
%[x^> y^>] p_3(\alpha) F(p_1(\alpha),p_2(\alpha);t),
%\end{equation*}
%and to decide whether equation~\eqref{eq:orbitEquation} implies that 
%\begin{equation*}
%xyF(x,y;t) = [x^> y^>] p(\alpha).
%\end{equation*}
%%when $\alpha$ is an element of $\overline{\mathbb{Q}(x,y)}$, given by its minimal polynomial over $\mathbb{Q}[x,y]$, and~$p_1(\alpha),\ p_2(\alpha)$ and $p_3(\alpha)$ are polynomials in $\alpha$ over $\mathbb{Q}(x,y)$. 
%A priori the application of $[x^>]$ and $[y^>]$ only makes sense for a series whose coefficients are Laurent ponomials in $x$ and $y$, respectively. Its positive part is the series which results from discarding all terms which involve a non-positive power of $x$ or $y$, respectively. 
%We did not stress this point in Example~\ref{ex:OSmethod} because the right hand side of the orbit equation can be unambiguously understood as an element of~$\mathbb{Q}[x,\bar{x},y,\bar{y}][[t]]$ whose positive part with respect to $x$ and $y$ is well-defined. We note that in~\cite{Rika} the problem was discussed when $p_1(\alpha), p_2(\alpha)$ and $p_3(\alpha)$ are rational functions. We begin with a simple example, and state the problem more precisely afterwards.
%
%\begin{Example}
%It is ambiguous to speak of the series solution $Y$ of the equation 
%\begin{equation*}
%(1-x)Y-1=0
%\end{equation*}
%as it depends on the field of Laurent series over which it is solved, and so does its positive part with respect to $x$.
%While in $\mathbb{Q}((x))$ the solution is $Y = \sum_{k=0}^\infty x^k$ and its positive part is $[x^>] Y = \sum_{k=1}^\infty x^k$, in $\mathbb{Q}((\bar{x}))$ it is $Y = -\sum_{k=1}^\infty \bar{x}^k$ and $[x^>] Y = 0$.
%
%When the equation is solved over $\mathbb{Q}(x)$, its solution is $Y=1/(1-x)$. In order to define the positive part of such a rational function, we need to associate a series to it. There are two options according to the choice which of the two terms in the denominator of $1/(1-x)$ is considered as the leading term.
%%, a rational function, whose positive part, of course, is not defined. But $Y$ can be expanded as a geometric series, and there are two ways of doing so. The denominator of $Y$ is $1-x$, its leading term is either $1$ or $-x$, depending on the term ordering. 
%We will see below that the choice of the term ordering induces an embedding of $\mathbb{Q}(x)$ into a field of Laurent series. For the moment we just note that the choice of $1$ as the leading term of $1-x$ corresponds to considering $\mathbb{Q}(x)$ as a subfield of $\mathbb{Q}((x))$, while the choice of $-x$ corresponds to considering it as a subfield of $\mathbb{Q}((\bar{x}))$. In the first case~$Y = 1/(1-x)$ expands to $Y=\sum_{k=1}^{\infty} x^k$, in the second case to $Y=\sum_{k = 1}^{\infty} \bar{x}^k$.
%\end{Example}
%
%We now state the problem of this section more precisely by formulating three questions. Let $\{m_1,\dots,m_n\}$ be a set of polynomials over $\mathbb{Q}(x,y)$, let $m(X)$ be the minimal polynomial of a primitive element $\alpha$ of their splitting field, and let $p_1(\alpha),\dots,p_m(\alpha)$ be representations of their roots as polynomials in $\alpha$ over $\mathbb{Q}(x,y)$ as output by Algorithm~\ref{alg:5}. The questions are:
%\begin{enumerate}
%\item What are the fields of Puiseux series $\mathbb{Q}(x,y)[\alpha]$ can be embedded into?
%\item What are the series that correspond to $\alpha$ and $p_1(\alpha),\dots,p_m(\alpha)$?
%\item What are their supports?
%\end{enumerate}
%
%An answer to the first question is found in~\cite{Manuel}, an exposition of a theory of Laurent series in several variables, and in~\cite{MacDonald}, a discussion of a (generalized) Newton-Puiseux algorithm. We will see that the latter also answers the other questions. We first collect some facts about Laurent series, and explain under which condition the product of two Laurent series is well-defined. The reader is referred to~\cite{Manuel} for details.
%
%\begin{Definition}
%A subset $C$ of $\mathbb{R}^n$ is called a cone if $\lambda C = C$ for every $\lambda\in\mathbb{R}_{\geq 0}$. It is called a polyhedral cone if there are $v_1,\dots,v_k \in \mathbb{R}^n$ such that $C = \mathbb{R}_{\geq 0} v_1 + \dots + \mathbb{R}_{\geq 0} v_k$, and rational if $v_1,\dots,v_k$ can be chosen to be elements of $\mathbb{Q}^n$. A cone $C$ is called convex if $\lambda v + (1-\lambda) w\in C$ for all $v,w\in C$ and all $\lambda\in[0,1]$, and strictly convex if, in addition, $C\cap (-C) = \{0\}$. The dual $C^*$ of $C$ is $C^*=\{ u\in\mathbb{R}^n \mid \langle u, C\rangle \leq 0 \}$.
%\end{Definition}
%
%\begin{Proposition}\label{prop:intDom}
%Let $C$ be a strictly convex cone of $\mathbb{R}^n$. Then
%\begin{equation*}
%\mathbb{C}_C[[x]] := \left \{ \phi = \sum_{I\in\mathbb{Q}^n} \phi_I x^I \;\middle|\; \phi_I \in\mathbb{C} \text{ and } \mathrm{supp} (\phi) \subseteq C\cap \frac{1}{k}\mathbb{Z}^n \text{ for some } k \right\}
%\end{equation*}
%forms an integral domain with respect to addition and multiplication.
%\end{Proposition}
%\begin{Proposition}
%Let $\preceq$ be an additive total order on $\mathbb{Q}^n$, i.e. a total order on $\mathbb{Q}^n$ that is compatible with the addition on $\mathbb{Q}^n$. Let $\mathcal{C}$ be the set of all cones whose smallest element with respect to $\preceq$ is $0$, and define 
%\begin{equation*}
%\mathbb{C}_{\preceq}[[x]] := \bigcup_{C\in \mathcal{C}} \mathbb{C}_C[[x]] \quad \text{and} 
%\quad \mathbb{C}_{\preceq}((x)) :=\bigcup_{e\in\mathbb{Q}^n} x^e \mathbb{C}_{\preceq} [[x]].
%\end{equation*}
%Then $\mathbb{C}_{\preceq}[[x]]$ is a ring, and $\mathbb{C}_{\preceq}((x))$ is a field.
%\end{Proposition}
%
%We note that any element $w\in\mathbb{R}^n$ whose components are linearly independent over $\mathbb{Q}$ defines an additive total order $\preceq$ on $\mathbb{Q}^n$ by
%\begin{equation*}
%v_1\preceq v_2 \quad :\Longleftrightarrow \quad \langle v_1,w\rangle \geq \langle v_2,w\rangle.
%\end{equation*}
%We also note that not every additive total order on $\mathbb{Q}^n$ is of this type. Consider the lexicographic order $\preceq_{lex}$ on $\mathbb{Q}^n$, for instance. A full classification of additive total orders on $\mathbb{Q}^n$ is given in~\cite[Theorem 4]{robbiano1985}. For convenience of the reader we state it here.
%
%\begin{Definition}
%Let $v$ be an element of $\mathbb{R}^n$. The rational dimension of $v$, denoted by $\mathrm{d}(v)$, is the dimension of the $\mathbb{Q}$-vector space generated by the components of $v$.
%\end{Definition}
%\begin{Theorem}
%For any additive total order $\preceq$ on $\mathbb{Q}^n$, there exist non-zero pairwise orthogonal vectors $u_1,\dots,u_s\in\mathbb{R}^n$ such that $\mathrm{d}(u_1) + \dots + \mathrm{d}(u_s) = n$ and
%\begin{equation*}
%\iota: (\mathbb{Q}^n, \preceq) \rightarrow (\mathbb{R}^n, \preceq_{lex}) \quad \text{defined by} \quad \iota(v) = (v\cdot u_1,\dots, v\cdot u_s)
%\end{equation*}
%is an injective order homomorphism.
%\end{Theorem}
%In the following we assume, without loss of generality, that the total orders we are working with are induced by elements of $\mathbb{R}^n$ whose components are linearly independent over $\mathbb{Q}$. 
%
%Given an additive total order $\preceq$ on $\mathbb{Q}^2$, the generalized Newton-Puiseux algorithm~\cite{MacDonald} implies that $\mathbb{C}_{\preceq}((x,y))$ is algebraically closed. Consequently, the splitting field $\mathbb{Q}((x,y))[\alpha]$ of any set $\{m_1,\dots,m_n\}$ of polynomials over $\mathbb{Q}(x,y)$ can be viewed as a subfield of $\mathbb{C}_{\preceq}((x,y))$. This settles the first question. 
%
%A look at the construction of multiplicative inverses of elements of~$\mathbb{C}_{\preceq}((x,y))$ settles the other two questions for elements of $\mathbb{Q}(x,y)$. Let $p(x,y)$ be a polynomial in $\mathbb{Q}[x,y]$ and let $\preceq$ be an additive total order on $\mathbb{Q}^2$. Then $p(x,y)$ has a leading term $tm$ with respect to $\preceq$ and the support of the Laurent polynomial $p(x,y)/tm$ is contained in a strictly convex cone $C$ which has a minimal element with respect to $\preceq$. Since $p(x,y)/tm$ is an element of $\mathbb{Q}_C[[x,y]]$, and since its constant term is $1$, by~\cite[Theorem 12]{Manuel} it has a multiplicative inverse in $\mathbb{Q}_C[[x,y]]$. Since $\mathbb{Q}_C[[x,y]]$ is an integral domain, as noted in Proposition~\ref{prop:intDom}, this multiplicative inverse is unique, in particular it is independent of the choice of $\preceq$ as long as the leading term of $p(x,y)$ is $tm$. The terms $tm$ of $p(x,y)$ for which $p(x,y)/tm$ is contained in a strictly convex cone are exactly those terms which correspond to vertices of the convex hull of the support of $p(x,y)$, and each choice of such a term gives rise to a multiplicative inverse of $p(x,y)$. These multiplicative inverses can also be seen as geometric series expansions of the rational function $1/p(x,y)$, and it is not difficult to see that these series expansions are pairwise different. 
%
%\begin{Example}\label{ex:inv}
%Consider the polynomial $p(x,y) = 1 - x - y$. The convex hull of the support of $p(x,y)$ has three vertices, $(0,0)$, $(1,0)$ and $(0,1)$. Consequently, there are three series, which qualify as multiplicative inverses, given by the geometric series $\sum_{k=0}^\infty (x+y)^k$ and $-\bar{x} \sum_{k=0}^\infty (\bar{x}-\bar{x}y)^k$ and $-\bar{y} \sum_{k=0}^\infty (\bar{y}-x\bar{y})^k$. Their supports are contained in the (shifted) cones~$\langle (1,0),(0,1)\rangle$, and $(-1,0) + \langle (-1,0),(-1,1)\rangle$ and~$(0,-1) + \langle (1,-1),(0,-1) \rangle$, respectively.
%\end{Example}
%
%We summarize the consequences of these observations in the next proposition.
%
%\begin{Proposition}
%Given two non-zero elements $p(x,y)$ and $q(x,y)$ of $\mathbb{Q}[x,y]$, there is a bijection between the series solutions $X$ of 
%\begin{equation*}
%p(x,y) - q(x,y) X = 0
%\end{equation*}
%and the vertices of the convex hull of the support of $q(x,y)$. For a term $tm$ whose exponent vector is a vertex of the convex hull of the support of $q(x,y)$ the corresponding series solution is the product of $p(x,y)$ and the geometric series expansion of $\frac{1}{tm}\frac{1}{1-\left(1-\frac{q}{tm}\right)}$, i.e. the series
%\begin{equation*}
%X(x,y) = p(x,y)\frac{1}{tm} \sum_{k=0}^\infty \left(1-\frac{q}{tm}\right)^k.
%\end{equation*}
%Its support is contained in a shift of the cone generated by the support of $1-q/tm$.
%\end{Proposition}
%
%To construct series solutions of polynomial equations over $\mathbb{Q}(x,y)$, which in general are not series expansions of rational functions, we use a generalized Newton-Puiseux algorithm~\cite{MacDonald}. See also~\cite{VanderWaerden} for the classical Newton-Puiseux algorithm. 
%
%The generalized Newton-Puiseux algorithm answers the second of the three questions of this section. By construction of $m(X)$ and $p_1,\dots,p_m$, any series $\phi$ that is a root of $m(X)$, and any total order $\preceq$ on $\mathbb{Q}^2$ that is compatible with $\phi$, allow to identify the splitting field $\mathbb{Q}(x,y)[\alpha]$ of $\{m_1,\dots,m_n\}$ over $\mathbb{Q}(x,y)$ with $\mathbb{Q}(x,y)[\phi]$ in $\mathbb{C}_{\preceq}((x,y))$. In particular, the roots of $m_1,\dots, m_n$ in $\mathbb{C}_{\preceq}((x,y))$ are the series that correspond with $p_1(\alpha),\dots,p_m(\alpha)$.
%
%For determining the support of compositions of series we make use of the next theorem. It gives a sufficient condition for the composition of Laurent series to be well-defined, and provides a cone containing its support in that case. Its proof can be found in~\cite{Manuel}.
%
%\begin{Theorem}\label{theorem:comp}
%Let $C\subseteq\mathbb{R}^n$ be a strictly convex cone and $F\in\mathbb{C}_C[[x_1,\dots,x_n]]$, and let~$\preceq$ be an additive order on $\mathbb{Z}^m$ and $g_1,\dots,g_n\in\mathbb{C}_{\preceq}((y_1,\dots,y_m))\setminus \{0\}$. Let $M\in\mathbb{Z}^{m\times n}$ be the matrix whose $i$-th column consists of the leading exponent of $g_i(y_1,\dots,y_m)$, and let $C'$ be a cone that contains the image $MC$ of $C$ under $M$ and $\mathrm{supp}(g_i / \mathrm{lt}(g_i))$ for $i=1,\dots,n$. If $C \cap \mathrm{ker} M = \{0\}$ and if $C'$ is strictly convex, then $F(g_1,\dots,g_n)$ is well-defined and belongs to $\mathbb{C}_{C'}[[y_1,\dots,y_m]]$.
%\end{Theorem}
%
%\begin{Example}\label{ex:comp0}
%Let $F(x,y;t)$ be a series in $\mathbb{Q}[x,y][[t]]$ whose support is contained in $C = \langle (1,0,1), (0,1,1), (1,1,1), (0,0,1) \rangle$, and 
%let $\preceq$ be any extension of the additive order from Example~\ref{ex:NPA} to $\mathbb{Q}^3$ and $g\in\mathbb{C}_{\preceq}((x,y,t))$ one of the series constructed there. Recall that its leading exponent was $(0,1,0)$ and that its support was contained in $(0,1,0)+\langle (1,1,0), (2,-1,0) \rangle$. To see that $F(g(x,y),x,t)$ is well-defined, observe that the kernel of the matrix $M$ of leading exponents of $g, x$ and $t$ is spanned by $(1,-1,0)$ and that its intersection with $C$ is trivial. The cone $C'$ generated by $MC = \langle (0,2,1), (0,0,1) \rangle$ and $(1,1,0)$ and $(2,-1,0)$ is strictly convex. Therefore,~$F(g(x,y),x,t)$ is well-defined and an element of $\mathbb{C}_{C'}[[x,y,t]]$.
%\end{Example}
%
%We are interested in applying Theorem~\ref{theorem:comp} when $F(x,y;t)\in\mathbb{C}[x,y][[t]]$ is the generating function of a model of walks, $g_1,g_2\in\mathbb{C}_{\preceq}((x,y,t))$ represent the components of an element of its orbit, and $g_3 = t$. The next lemma states that in this case the assumptions of the theorem are always fulfilled.
%
%\begin{Lemma}
%Let $C$ be a strictly convex cone in $\mathbb{R}^3$ such that $C\cap \left(\mathbb{R}^2\times \{0\}\right) = \{0\}$, and let $F\in\mathbb{C}_C[[x,y,t]]$. Let $\preceq$ be an additive total order on $\mathbb{Q}^3$ and $g_1,g_2\in\mathbb{C}_{\preceq}((x,y,t))$ be independent of $t$, and let $M$ be the matrix whose columns are the leading exponents of $g_1,g_2$ and $t$. Then $C\cap \mathrm{ker}(M) = \{0\}$, and the cone generated by $MC$ and $\mathrm{supp}(g_i/\mathrm{lt}(g_i))$, $i\in\{1,2\}$ is strictly convex.
%\end{Lemma}
%\begin{proof}
%The series $g_1$ and $g_2$ do not depend on $t$, therefore $\mathrm{ker}(M) \subseteq \mathbb{R}^2\times \{0\}$, and so~$C\cap \mathrm{ker}(M) = \{0\}$, by assumption on $C$. Since $g_1$ and $g_2$ are elements of $\mathbb{C}_{\preceq}((x,y))$, the cone generated by the support of $g_1/\mathrm{lt}(g_1)$ and $g_2/\mathrm{lt}(g_2)$ is strictly convex, and because $g_1$ and $g_2$ are independent of $t$, it is contained in $\mathbb{R}^2\times \{0\}$. The shape of $M$ implies $MC \cap\left( \mathbb{R}^2\times \{0\}\right) = M\left(C \cap\left( \mathbb{R}^2\times \{0\}\right) \right) =  \{0\}$. To finish the proof of the lemma, it is therefore sufficient to show that $MC$ is strictly convex. Assume that there is a~$v\neq 0$ such that $v\in MC$ and $-v\in MC$. Then there are $u_1,u_2\in C$ such that~$Mu_1=v$ and $Mu_2=-v$. But then $M(u_1+u_2) = 0$, i.e. $u_1+u_2\in\mathrm{ker}(M)$. Together with~$u_1+u_2\in C$ and $C\cap\mathrm{ker}(M) = \{0\}$ this implies that $u_1 + u_2 = 0$. Since~$C$ is strictly convex, $u_1=0=u_2$, and therefore $v=0$. So $MC$ is strictly convex as well.
%\end{proof}
%
%\begin{Remark}
%In general, the image of a strictly convex cone under a linear map need not be strictly convex. Consider e.g. the image of $\mathbb{R}_{\geq 0}^2$ under $(x,y) \mapsto x-y$.
%\end{Remark}
%
%Taking into account how $C'$ was constructed in Theorem~\ref{theorem:comp}, it is not surprising that in general it does not need to be minimal. Here is an example.
%
%\begin{Example}\label{ex:non-min}
%Let $m(x,y,z) = 4x^2y+(x^2y+xy^2+xy+y)^2-z^2$, and let $C$ be the cone generated by its support, i.e. $C = \langle (0,1,0), (0,0,1),(2,1,0) \rangle$. Let again $\preceq$ be any extension of the additive order from Example~\ref{ex:NPA} to $\mathbb{Q}^3$ and $g\in\mathbb{C}_{\preceq}((x,y,t))$ one of the series constructed there. According to Theorem~\ref{theorem:comp} the composition $m(x,y,g(x,y))$ is well-defined and its support is contained in $C'=\langle (1,1,0),(2,-1,0) \rangle$. But $C'$ is not minimal, since $m(x,y,g(x,y)) = 0$, by construction of $g$. 
%\end{Example}
%
%In case the composition is algebraic and we know its minimal polynomial, we can also use the Newton-Puiseux algorithm to determine a (shifted) cone that contains its support. To explain how, let us draw our attention to another aspect of the Newton-Puiseux algorithm. Algorithm~\ref{alg:NPA} not only allows to determine term by term the series solutions of a polynomial equation, but also to encode an algebraic series by its minimal polynomial, a total order, and the first terms of the series with respect to this total order. This is a consequence of Proposition\ref{prop:uniqueness}.
%
%We finally explain, by summarizing what has just been presented, how to perform the extraction of the positive part of the left hand side of an orbit equation
%\begin{equation*}
%\sum_i p_{i,3}(\alpha) F(p_{i,1}(\alpha),p_{i,2}(\alpha);t).
%\end{equation*}
%Recall that $\alpha$ is a primitive element for the splitting field of the minimal polynomials of the elements of the orbit, and that $p_{i,1}(X),p_{i,2}(X)$ and $p_{i,3}(X)$ are elements of $\mathbb{Q}(x,y)[X]$ such that $(p_{i,1}(\alpha),p_{i,2}(\alpha))$ represent the elements of the orbit and the right hand side of the orbit equation does not involve $F$. We assume that~$(p_{1,1}(\alpha),p_{1,2}(\alpha)) = (x,y)$ and that $p_{1,3}(\alpha) = xy$. When we apply $[x^>y^>]$ to the above expression, we want to know whether the only term that remains is $xyF(x,y)$, i.e. whether $[x^>y^>]p_{i,3}(\alpha) F(p_{i,1}(\alpha),p_{i,2}(\alpha))$ is zero for all $i>1$. In this section we have explained how $\mathbb{C}(x,y)[\alpha]$ can be viewed as a subfield of a field $\mathbb{C}_{\preceq}((x,y))$ of Laurent series, and how, for any element $p(\alpha)$ of $\mathbb{C}(x,y)[\alpha]$, Algorithm~\ref{alg:NPA} allows to determine its leading exponent as well as a (shifted) cone that contains its support. Based on this, Theorem~\ref{theorem:comp} can be used to determine a cone that contains the support of $F(p_{i,1}(\alpha),p_{i,2}(\alpha))$, and consequently, a shifted cone that contains the support of $p_{i,3}(\alpha) F(p_{i,1}(\alpha),p_{i,2}(\alpha))$.
%
%\begin{Example}
%We show that the generating function of walks in $\mathbb{N}^2$ that start at~$(0,0)$ and take their steps alternatingly from $S_0 = \{(-1,1),(0,0),(0,1),(1,0)\}$ and~$S_1=\{(-1,-1),(0,0)\}$ is D-finite. The Minkowski-sum $S_0+S_1$ of $S_0$ and $S_1$ can not be associated with a group. But it can be associated with a finite orbit. Its elements are $(x,y),(x,\bar{y})$, $(p_1(\alpha),y)$ and $(p_{-1}(\alpha),y)$, and $(p_1(\alpha),\bar{y})$ and $(p_{-1}(\alpha),\bar{y})$, where
%\begin{equation*}
%p_i(X) = \frac{x + y + x y + x y^2 + i X}{2 x^2 y} \quad \text{and} \quad \alpha = \sqrt{4 x^3 y^2 + (x + y + x y + x y^2)^2}.
%\end{equation*}
%We consider their components as elements of the extension of $\mathbb{C}(x,y)$ by a root $\alpha$ of
%\begin{equation*}
%m(X) = X^2 - 4 x^3 y^2 - (x + y + x y + x y^2)^2.
%\end{equation*}
%Plugging the elements of the orbit into one of the kernel equations, for instance into
%\begin{equation*}
%(1-t^2S_0 S_1) F_0 = 1 - t\bar{x}\bar{y} (F_1(x,0) + F_1(0,y) - F_1(0,0)) - t^2(\bar{x}\bar{y}+1)\bar{x}y F_0(0,y),
%\end{equation*}
%forming a linear combination of the resulting equations with undetermined coefficients, and equating the  coefficients of the sections of $F_0$ and $F_1$ to zero results in a linear system over $\mathbb{C}(x,y)[\alpha]$. The vector space of solutions is $1$-dimensional, and so is the vector space of section-free orbit equations. The latter is generated by the equation
%\begin{equation*}
%F_0(x,y) - \bar{y}^2 F_0(x,\bar{y}) - \sum_{i,j = \pm1} c_{ij}(\alpha) F_0(p_i(\alpha),y^j) = \frac{(-1 + y^2) (2 y - x^3 y + x (1 + y + y^2))}{x^3 y^3(1 - t^2 S_0 S_1)}.
%\end{equation*}
%The coefficients $c_{ij}(\alpha)$ in $\mathbb{C}(x,y)[\alpha]$ are
%\begin{align*}
%c_{ij}(\alpha) = i \frac{x + 2 y + x y + x y^2}{2 x^3 y} + j \frac{\alpha (2 y^2 + 2 x^3 y^2 + 3 x y (1 + y + y^2) + x^2 (1 + y + y^2)^2)}{2 x^3 y (y^2 + 4 x^3 y^2 + 2 x y (1 + y + y^2) + x^2 (1 + y + y^2)^2)}.
%\end{align*}
%Let $\preceq$ be the total order on $\mathbb{Q}^2$ defined by $w = (\sqrt{2},1/2)$ and let $\phi$ be the series solution of $m(X) = 0$ in $\mathbb{C}_{\preceq}((x,y))$ whose first term is $2x^{3/2}y$.
%% and whose support is contained in $(3/2,1)+ \langle (-1,2),(-1,-2) \rangle$. 
%We identify $p_i(\alpha)$ and~$c_{ij}(\alpha)$ with $p_i(\phi)$ and $c_{ij}(\phi)$ in $\mathbb{C}_{\preceq}((x,y))$ and show that the only term on the left hand side of the equation that remains when applying $[x^{\geq}y^{\geq}]$ is $F_0(x,y)$. Consequently, $F_0$ is the non-negative part of a rational function, and therefore D-finite. Obviously, $[x^{\geq}y^{\geq}] F_0(x,y) = F_0(x,y)$ and $[x^{\geq}y^{\geq}] \bar{y}^2F_0(x,\bar{y}) = 0$. As in Example~\ref{ex:comp}, one can show that 
%\begin{equation*}
%\mathrm{supp}(p_i(\phi)) \subseteq (-1/2,0,0) + \langle (-1,2,0),(-1,-2,0) \rangle
%\end{equation*}
%and 
%\begin{equation*}
%\mathrm{supp}(c_{ij}(\phi)) \subseteq (-3/2,-1+j,0) + \langle (-1,2,0),(-1,-2,0) \rangle,
%\end{equation*}
%and as in Example~\ref{ex:comp0} one can show that
%\begin{equation*}
%\mathrm{supp}(F_0(p_i(\phi),y^j))\subseteq \langle (0,0,1),(0,j,1),(-1,2,0),(-1,-2,0) \rangle.
%\end{equation*}
%Therefore, 
%\begin{equation*}
%\mathrm{supp}(c_{ij}(\phi)F_0(p_i(\phi),y^j)) \subseteq (-3/2,-1+j,0) + \langle (0,0,1),(0,j,1),(-1,2,0),(-1,-2,0) \rangle,
%\end{equation*}
%and so 
%\begin{equation*}
%[x^{\geq}y^{\geq}] c_{ij}(\phi) F_0(p_i(\phi),y^j) = 0.
%\end{equation*}
%\end{Example}
%
%We close this section with the statement of an algorithm which can be used to extract the positive part of the left-hand side of an orbit equation and some remarks.
%
%\begin{Algorithm}\label{alg:ppe}
%Input: An irreducible polynomial $m(X)$ over $\mathbb{Q}[x,y]$, a list $L$ of tuples $(p_1,p_2,p_3)$ of polynomials over $\mathbb{Q}(x,y)$, and a cone $C$ that contains the support of a series $F(x,y)$ such that $C\cap\left(\mathbb{R}^2\times\{0\}\right) = \{0\}$.\\
%Output: True, if there is a total order $\preceq$ on $\mathbb{Q}^3$ and a series solution $\phi$ of $m(X)=0$ in $\mathbb{C}_{\preceq}((x,y))$ such that $[x^{\geq} y^{\geq}]p_3(\phi) F(p_1(\phi),p_2(\phi)) = 0$ for all $(p_1,p_2,p_3)$ in $L$; otherwise Failed.
% \step 10 For each series solution $\phi$ of $m(X) = 0$, do: 
% \step 21 Determine a list $L_1$ of strictly convex cones such that for each cone $C_1\in L_1$ there is some $\alpha\in\mathbb{Q}^3$ such that $\mathrm{supp}(\phi)\subseteq \alpha + C_1$.
% \step 31 Compute a list $L_2$ of (strictly convex) cones $C_2$ such that for every polynomial $p(X)$ which appears as a component of an element of $L$ and every non-negative integer $i$, there is an expansion of $[X^i]p(X)$ whose support is contained in $\alpha + C_2$ for some $\alpha\in\mathbb{Q}^3$.
% \step 41 For each $C_1\in L_1$ and for each $C_2\in L_2$ such that $C_1 + C_2$ is strictly convex, do:
% \step 52 Choose a total order $\preceq$ on $\mathbb{Q}^3$ that is compatible with $C_1+C_2$, and determine for each $p(X)$ which appears as a component of an element of $L$ a list $L_p$ of pairs $(\alpha_p,C_p)$ such that $\alpha_p$ is a vertex of the convex hull of the support of $p(\phi)$ in $\mathbb{C}_{\preceq}((x,y))$ and $C_p$ is a strictly convex cone such that its support is contained in $\alpha_p + C_p$.
% \step 62 If for each $(p_1,p_2,p_3)$ in $L$ there are elements $(\alpha_{p_i},C_{p_i})$ of $L_{p_i}$ such that for the cone $\tilde{C}$ computed from $C$ and $(\alpha_{p_1},C_{p_1})$ and $(\alpha_{p_2},C_{p_2})$ using Theorem~\ref{theorem:comp} we have
% \begin{equation*}
% \left( \mathbb{Q}_{\geq 0}^2\times \mathbb{Q}\right) \cap \left(\alpha_{p_3} + C_{p_3} + \tilde{C} \right)= \emptyset,
% \end{equation*}
% then return True.
%  \step 70 Return Failed. 
%\end{Algorithm}
%
%\begin{Remark}
%The series solutions $\phi$ over which Algorithm~\ref{alg:ppe} loops in step~$1$ are supposed to be computed using the Newton-Puiseux algorithm, and so are the cones in step~$2$. The cones in step~$3$ can be determined simply by investigating the possible geometric series expansions of the rational functions that appear as coefficients of the polynomials in $L$. The strictly convex cones $C_1+C_2$ over which the loop in step~$4$ iterates are constructed such that any element $w$ of the dual of~$(C_1+C_2)^*$ that induces a total order $\preceq$ on $\mathbb{Q}^3$ induces a series expansion of $p(\phi)$ in~$\mathbb{C}_{\preceq}((x,y,t))$, for every polynomial $p(X)$ which appears as a component of an element of $L$, that does not depend on the specific choice of $w$ in $(C_1+C_2)^*$. Therefore, also the pairs~$(\alpha_p,C_p)$ determined in step~$5$, again using the Newton-Puiseux algorithm, do not depend on the specific choice of $w$ in $(C_1+C_2)^*$. 
%\end{Remark}
%
%\begin{Remark}
%In general, the vector space of orbit equations of a model need not be of dimension $1$. If its dimension is $2$ or higher it is not clear which orbit equation should be chosen to perform the extraction of its positive part. This degree of freedom for choosing the input of Algorithm~\ref{alg:ppe} can be reduced by determining the list of $F(p_1(\alpha),p_2(\alpha))$ for which $[x^{\geq}y^{\geq}] p_3(\alpha) F(p_1(\alpha),p_2(\alpha))$ is different from $0$, regardless of the field $\mathbb{C}_{\preceq}((x,y,t))$ of Laurent series $p_1(\alpha)$ and $p_2(\alpha)$ are considered to be elements of, and regardless of the choice of $p_3(\alpha)$ in $\mathbb{C}_{\preceq}((x,y,t))\setminus\{0\}$. One can then compute the vector space of orbit equations which does not involve any of these terms, and in case it has dimension $1$, apply Algorithm~\ref{alg:ppe}.
%\end{Remark}
%
%\begin{Remark}
%It is natural to ask whether it can happen that Algorithm~\ref{alg:ppe} returns Failed although the only term on the left-hand side of the orbit equation that remains when applying $[x^{\geq}y^{\geq}]$ is $F(x,y;t)$. In~\cite[Proposition~$24$]{large} it was shown that it can happen there are two terms $p_3(\alpha)F(p_1(\alpha),p_2(\alpha);t)$ whose expansions involve terms with non-negative powers in $x$ and $y$, although their sum does not. Whether this is the only reason depends on whether the (shifted) cone computed in Algorithm~\ref{alg:NPA} which contains the support of a series solution of a polynomial equation is minimal or not, see Conjecture~\ref{conj:minCone}.
%\end{Remark}
%
%In many cases we cannot apply $[x^> y^>]$ to a section-free orbit equation to find an expression for its generating function. In some cases the orbit is not finite and in others the orbit is finite but there is no section-free orbit-equation. In some cases the section-free orbit-equations only have an orbit-sum that is equal to $0$, and again for others the orbit-sum is different from $0$ but the dimension of the vector space formed by these equations is greater than $1$ and it is not clear which equation should be chosen to extract the positive part. 
%
%
%\bibliographystyle{plain}
%\bibliography{orbitSumMethod}
\end{document}
